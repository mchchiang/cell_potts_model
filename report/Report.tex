%
%                       This is a LaTeX 2e version of the
%                       laboratory project template file.
\documentclass[a4paper,12pt]{article}
\usepackage{fullpage,epsf}
\usepackage{amsmath}
\usepackage{amsfonts}
\usepackage{amssymb}
\usepackage{amstext}
\usepackage{bm}
\usepackage{braket}
\usepackage{array}
\usepackage{graphicx}
\usepackage{tabularx}
\usepackage{url}
\usepackage{verbatim}
\usepackage{listings}
\usepackage{color}
\usepackage{courier}
\usepackage{epstopdf}
\usepackage{placeins}
\usepackage{subcaption}
\usepackage[font=footnotesize]{caption}
\usepackage[font=footnotesize]{subcaption}
\epstopdfsetup{update}
\usepackage[]{cite}


%%%%%%%%%%%%%%%%%%%%%%%%%%%%%%%%%%%%%%%%%%%%%%%%%%%%%
\renewcommand{\vec}[1]{\mathbf{#1}}
\newcommand{\abs}[1]{\left|#1\right|}
\newcommand{\inc}{\Delta}
\newcommand{\etal}{\emph{et al. }}
%%%%%%%%%%%%%%%%%%%%%%%%%%%%%%%%%%%%%%%%%%%%%%%%%%%%%

%
%                       This section generates a title page
%                       Edit only the sections indicated to put
%                       in the project title, your name, supervisor,
%                       project length in weeks and submission date
%
\begin{document}
\pagestyle{empty}                       % No numbers of title page                      
\epsfxsize=40mm                         % Size of crest
\begin{minipage}[b]{110mm}
        {\Huge\bf School of Physics\\ and Astronomy
        \vspace*{17mm}}
\end{minipage}
\hfill
\begin{minipage}[t]{40mm}               
        \makebox[40mm]{
        \includegraphics[width=4cm]{crest.jpg}}
\end{minipage}
\par\noindent                                           % Centre Title, and name
\vspace*{2cm}
\begin{center}
        \Large\bf \Large\bf Senior Honours Project\\
        \Large\bf Physics 4\\[10pt]                     % Change to MP/CP/Astro
        \LARGE\bf Investigation of the \\Jamming Transition in Cellular Potts Model       % Change to suit
\end{center}
\vspace*{0.5cm}
\begin{center}
        \bf Michael Chiang\\                           % Repace with your name
        25 March 2015                                    % Submission Date
\end{center}
\vspace*{5mm}
%
%                       Insert your abstract HERE
%                       
\begin{abstract}
        The abstract is a short, concise explanation of the project
        covering the aims, outlines of techniques used and a short
        summary of the results. It should contain enough information to
        make the aims and success of the project clear, but contain no details.
        A typical abstract should be between 50 and 100 words.
\end{abstract}

\vspace*{1cm}

\subsubsection*{Declaration}

\begin{quotation}
        I declare that this project and report is my own work.
\end{quotation}

\vspace*{2cm}
Signature:\hspace*{8cm}Date: 25 March 2015

\vfill
{\bf Supervisor:} Prof. D. Marenduzzo                 % Change to suit
\hfill
10 Weeks                                         % Change to suit
\newpage
%
%                       End of Title Page
\pagestyle{plain}                               % Page numbers at bottom
\setcounter{page}{1}                            % Set page number to 1
\tableofcontents                                % Makes Table of Contents

\break
\section{Introduction}
Cell monolayers are widely studied biophysical systems as they are a simple model of a tissue. They provide an economic approach to study complex biological processes that take place in multi-cellular organisms. A particular research area where cell monolayers have been used as a model is in the study of dense biological tissues. These tissues, where cells are closely packed against each other with minimal spacing in between, are often found in important physiological and pathological processes such as embryonic development, wound healing, and cancer metastasis \cite{friedl2009}.  

Recent experiments have demonstrated that these dense tissues exhibit both fluid and glass-like behaviours \cite{angelini2010, schoetz2013}. This is reminiscent of the jamming transition observed in condensed matter, which describes the transition of matter with disordered structure from a fluid (unjammed) to rigid (jammed) state. This transition is widely studied in non-living systems such as colloidal dispersions, granular matter, foams, and emulsions. 

Computer simulations have been conducted to further understand the mechanics behind the fluid and rigid behaviours of tissues. To this end, multiple models have been developed to describe the dynamics of cells within tissues. A particular successful model is the vertex model, where cells are represented as a polygonal tiling of space and their dynamics are governed by a set of force equations on the vertices of the polygons \cite{nagai2001}. Recent computational studies indicate that the model exhibits a jamming transition in agreement with empirical observations, with cell shape and cell motility being the key parameters in determining the rigidity of the tissue \cite{bi2015density, bi2015motility}.

Another widely used model in simulating cell dynamics in tissues is the cellular Potts model. This model represents cells as domains on a lattice, and their movements are governed by minimising an energy function that characterises the cellular interactions. There, however, has not been simulations conducted to verify whether the model has a fluid to rigid transition. With this in mind, the aim of this project is to implement the cellular Potts model and to investigate whether varying the model parameters would reproduce the jamming transition observed in dense tissues. 

The remain sections of this paper are organised as follows: Section 2 provides a description of the cellular Potts model, the concept of jamming, and its connection to the dynamics in dense biological tissues. Section 3 outlines the implementation of the model and the experimental methods for investigating whether CPM can induce the jamming transition. Section 4 and 5 present and discuss the results obtained from the simulations. 


%The introduction section of the report should introduce the project in
%more detail than in the abstract. In particular it should present the
%motivation, the aims, outline of techniques used, and the scope of the project. 
%It should also contain references to similar work in the
%same field to put your work in the correct context.
%
%As a general rule, people reading the abstract and introduction alone
%should have a good idea of the material in the project, the techniques
%employed and the results obtained. A typical introduction should be
%about 1 page, (300-450 words). 


\section{Background and Theory}
\subsection{Cellular Potts Model}
Developed by Glazier and Graner\cite{graner1992}, the cellular Potts model (CPM)\footnote{The CPM is also known as the GGH model, naming after Glazier and Graner who first proposed it and Hogeweg who extended it for various biological modelling applications.} is a stochastic model widely used in simulating the motion of closely packed cells in a monolayer. The model is an extension of the Potts model used in statistical physics and takes into account of the physical characteristics and dynamics of biological cells. The model has successfully explained collective behaviours such as cell sorting \cite{graner1992} and streaming\cite{szabo2010}.

In the CPM, a cell monolayer is represented on a two dimensional lattice. Each lattice site is assigned with a non-negative integer spin $\sigma$ known as the cell index, where $\sigma \in [0,N]$ and $N$ is the total number of cells. Two lattice sites which share a side or a corner are considered as neighbours. A biological cell is denoted by a set of connected sites with the same spin (see figure \ref{fig:CPM}). $\sigma = 0$ is reserved for enumerating sites that are not part of any cell. Throughout this project, periodic boundary condition is applied such that the boundaries have no influence on the cells dynamics.

\begin{figure}[h]
\centering
\includegraphics[width=7.0cm]{figure/CPMDiagram.png}
\caption{A pictorial representation of the cellular Potts model. A cell is represented by a domain in the lattice, which is a set of interconnecting sites with the same spin value. Diagram taken from Glazier \etal, Single Cell Based Models in Biology and Medicine [cite].}
\label{fig:CPM}
\end{figure}

The dynamics of the cells are evolved using a modified version of the Metropolis algorithm: In the simulation, a series of elementary steps are taken with the goal of minimising an effective Hamiltonian $H$ that characterises the properties of cells and their interactions. Each step is an attempt to change the spin of a random lattice site $\bm{a}$ to that of a randomly chosen neighbour site $\bm{b}$, where $\sigma(\bm{a}) \neq \sigma(\bm{b})$. The probability of accepting a spin change, $p\left(\sigma(\bm{a}) \rightarrow \sigma(\bm{b})\right)$, is determined by the associated change in energy ($\inc H$): if $\inc H \leq 0$, the spin change is always accepted; if $\inc H > 0$, the spin change is accepted with a probability of $\exp(-\inc H / kT)$, where $k$ is the Boltzmann constant and $T$ is the effective temperature. These relations can be conveniently summarised by the following expression:
\begin{eqnarray}
\ln p\left(\sigma(\bm{a}) \rightarrow \sigma(\bm{b})\right) = \min\left[0,-\frac{\inc H}{kT}\right].
\end{eqnarray}
The exact form of the Hamiltonian will be discussed below. Since changing $k$ and $T$ is equivalent to multiplying the energy function by an irrelevant constant, it is conventional to set $k$ and $T$ to $1$ and express the energy in units of $kT$. As updating the entire lattice takes more steps with a larger system, the number of elementary steps taken is not a proper measure of time. It is conventional to define a unit time as $L^2$ number of spin change attempts, where $L$ is the linear dimension of the lattice. This is known as one Monte-Carlo step (MCS). One also defines the acceptance rate as the ratio of the number of spin copies accepted to the number of attempts made. This is important for an accurate comparison between the time scales used in simulations and those used in experiments\cite{sanz2010}.

As mentioned above, the system's dynamics are governed by an effective Hamiltonian that describes the physical behaviours of cells in tissues. This is given by:
\begin{eqnarray}
\label{eqn:hamiltonian}
H = \sum_{\langle{\bm{x}, \bm{x'}\rangle}} J\left(\sigma(\bm{x}), \sigma(\bm{x'})\right) + \lambda \sum_{i\,=\,1}^{N} \left(a_i - A_i\right)^2 - P \;\sum_{i\,=\,1}^{N} \bm{n}_i \cdot \bm{r}_i.
\end{eqnarray}
The first term accounts for the overall interfacial behaviours of cells. This includes cell-cell adhesion and the surface tension of each cell. The term is a sum over the interfacial energy for all pairs of neighbours in the lattice. $J$ describes the specific interfacial energy between two sites $\bm{x}$ and $\bm{x}'$. For simulating an ensemble of homogenous cells, $J$ is specified by the following values:
\begin{eqnarray}
\label{eqn:interfacial_energy}
J(\sigma,\sigma') = \left\{
	\begin {array}{ll}
		0 & \textrm{for $\sigma = \sigma'$}\\
		\alpha & \textrm{for $\sigma, \sigma' > 0$ and $\sigma \neq \sigma'$}\\
		\beta & \textrm{for $\sigma$ or $\sigma' = 0$},
	\end{array}
\right.
\end{eqnarray}
where $\alpha$ and $\beta$ are parameters for modelling the interfacial energy between two cells and the free boundary energy, respectively. Previous studies have found that the magnitudes of these parameters affect the cell shapes and the flexibility of the cell boundaries: small magnitudes produce long and rough boundaries with more active cell dynamics, while large magnitudes result in straight boundaries with less cellular motion \cite{szabo2010}.

The second term addresses the fact that cells are approximately the same size due to their incompressibility and resistance to height fluctuations. These aspects are modelled as an elastic constraint in the Hamiltonian. $a_i$ and $A_i$ are the current and target area of the $i$th cell. $\lambda$ sets the strength of this constraint and governs the degree of fluctuations of each cell's area.

The third term is proposed by Szabo \emph{et al.} in their study of collective streaming behaviours in cell monolayers \cite{szabo2010}. It captures the nature that cells can generate a motile force and move along a polarized direction. The term models this effect by favouring spin changes which result in the cells moving along their preferred direction. Within the term, $\bm{n}_i$ is a polarity vector that describes the preferred direction of the cell, $\bm{r}_i$ is the centre of mass of the cell, and $P$ is a free parameter associated with the strength of the motility (taken to be homogeneous). The specific mechanisms of how a cell generates a polarized motion and how the polarity changes over time remain poorly understood and are subject to intense investigations [cite]. A simple model, as suggested by Bi \emph{et al.} \cite{bi2015motility} and is implemented in this work, is to assume that the polarity vector undergoes rotational diffusion over time. If one defines the polarity angle of the $i$th cell ($\theta_i$) as the angle between the cell's polarity vector ($\bm{n}_i$) and a reference axis (taken to be the $x$-axis herein), the rotational diffusion process is governed by the following equations:
\begin{eqnarray}
\partial_t\theta_i(t) & = & \eta_i(t)\\
\langle{\eta_i(t)\eta_j(t')\rangle} & = & 2D_r\delta(t-t')\delta_{ij}
\end{eqnarray}
where $\eta_i$ is a noise function with zero mean and a variance of $2D_r$ with $D_r$ being the rotational diffusion coefficient. On a computer, the evolution of the polarity angle over time can be represented by:
\begin{eqnarray}
\label{eqn:computerRotateDiff}
\theta_i(t+\Delta t) = \theta_i(t) + \sqrt{2D_r\Delta t}\,\tilde\eta
\end{eqnarray}
where $\tilde\eta$ is a random number generated such that the last term reproduces the statistical mean and variance of the noise function. For simplicity, this can be achieved by generating a uniformly distributed random number between $\sqrt{3}\,[-1,1)$ , where the factor of $\sqrt{3}$ comes from normalisation\footnote{See Appendix \ref{} for a detailed derivation of this relation.}.

\subsection{Jamming Transitions in Biological Tissues}
In condensed matter, jamming describes the transition from a fluid to rigid state in which the structure of the material remains disordered. This occurs in various non-biological systems such as emulsions, foams, granular materials, and colloidal dispersions\cite{hecke2010}. For example, colloidal particles can jam and become rigid when the packing fraction is increased beyond a critical point, while foams can lose its ability to flow when the applied stress is lowered. The conditions for the transition are described by the ``jamming phase diagram,'' which was first proposed by Liu and Nagel (see figure \ref{fig:JammingPhaseDiagram}) \cite{liu1998}. The key parameters governing the transition are the temperature, the applied stress, and the density of the material. For instance, in the well-known glass transition, a fluid becomes solid-like but maintains a disordered structure when cooled. This corresponds to moving downwards along the temperature -- 1/density plane in the diagram. 

\begin{figure}[h]
\centering
\includegraphics[width=9.0cm]{figure/JammingPhaseDiagram.jpeg}
\caption{The jamming phase diagram proposed by Liu and Nagel. The glass transition corresponds to moving vertically in the temperature -- 1/density plane. For systems where temperature is irrelevant, the focus is on the 1/density -- stress plane. }
\label{fig:JammingPhaseDiagram}
\end{figure}

For living epithelial and endothelial tissues, it is clear that they have fluid-like properties. The cells within them do not have long range positional order and can slide pass each other. However, recent experiments have shown that these dense biological tissues also display glassy, rigid dynamics: Firstly, the tissues exhibit \emph{viscoelasticity} \cite{schoetz2013}. They behave like an elastic solid at short timescales but like a viscous fluid at longer timescales. The typical time for the system to transit from a solid to a fluid-like state is known as the relaxation time. Secondly, the trajectories of individual cells reflect the \emph{caging effect} \cite{schoetz2013}. This is the phenomenon in which individual elements of the system are trapped in a ``cage'' formed by their surrounding neighbours due to dense packing. The element can only escape from the cage under co-operative movements with its neighbours, which only happens on rare occasions when the energy fluctuations are large. Thirdly, the cells also demonstrate \emph{dynamic heterogeneity} \cite{angelini2010}. This refers to the strong disparity in the movements and forces between individual constituents of the system over space and time.  

The exhibition of both fluid and glass-like behaviours in dense tissues have prompted investigations on the jamming transition in these systems. A major focus is on identifying the relevant parameters, analogous to those in the jamming phase diagram, that govern the transition. To this end, both experimental and computational approach have been used. Computer simulations have been particularly useful as they allow one to vary cell properties systematically, which would be difficult to achieve in experiments. Multiple models have been developed and extensively studied for characterising cell dynamics in tissues. Two widely-used models are the vertex model and the CPM.  

\subsection{Motivation of the Study}
Recently, Bi and co-workers demonstrated that the vertex model and its variant, the Self-Propelled Voronoi (SPV) model, exhibit the jamming transition observed in confluent tissues, where there are no gaps between cells \cite{bi2015density, bi2015motility}. These models represent individual cells as polygons that fill a two dimensional space. The dynamics are controlled by a set of force equations on the vertices or, in the case of SPV, the centres of the polygons. Researchers have found that the rigidity of the system is governed by three parameters: the cell shape index, the strength of individual cell motile force, and the persistent time. The cell shape index is a dimensionless ratio of the preferred perimeter of the cell to (the square root of) its preferred cross-sectional area. It is influenced by factors such as cell-cell adhesion and the cells' surface tension. The persistent time is the characteristic time a cell would travel before changing its direction. In the limit of zero cell motility, the cell shape index alone can predict the transition point, with higher cell-cell adhesion lead to more fluid-like behaviours. When the cells are motile and have a longer persistent time, the 

A less extensive effort, however, has been made on investigating whether the jamming transition exists in the CPM. Although the work done by Kabla \cite{kabla2012} and Szab\'o \cite{szabo2010} have hinted such a possibility, there has not been a specific study conducted to analyse jamming in the CPM.  This As suggested by Bi \emph{et al.}, cellular interactions and cell motility are two key parameters in determining the transition point. These correspond well to the parameters $\alpha$ and $P$, respectively, in the CPM. In light of this, there are two objectives in this project: Firstly, to determine whether the varying model parameter $\alpha$, which models interfacial effects of cells such as surface tension and cell-cell adhesion, has the ability to induce a jamming transition. Secondly, to understand the effect of altering the strength of cell motility has on the rigidity of the system.


\section{Methodology}
The project is divided into two parts to achieve the objectives as discussed in the previous section. The first part focuses on implementing the cellular Potts model on a computer; the second part is devoted to changing the model parameters systematically to investigate whether the model exhibits the jamming transition.

\subsection{Implementation of the Cellular Potts Model}
\subsubsection{Programming Language and Software Used}
The CPM was implemented in Java as it is a robust, platform-agnostic language. Java also provides an easy-to-use framework for developing a graphical user interface, which is also produced in this project to visualise the cell motions. To facilitate the development process, the Eclipse integrated development environment was used as it has a comprehensive code browsing and debugging toolkit. In addition, the version control system, Git, was used to maintain the history of the development and to facilitate the synchronisation of the code across multiple platforms. The ANT automated build system was also employed in this project to allow rapid code compilation.

\subsubsection{Code Design and Structure}
The main structure and design of the code is described as follow: There are three main groups of classes in the program (see figure \ref{fig:ClassDiagram}). Firstly, there are the model classes that handle the main computation as specified by the CPM. These include the \texttt{CellPottsModel} and the \texttt{SpinModel} class. \texttt{CellPottsModel} contains most of the logic of the model, including the Metropolis algorithm for updating spins and the evaluation of the various statistical quantities measured in the experiments. The \texttt{SpinModel} is the main interface for which other classes access data from the model. The second group is the visualisation classes which display the model to the user in a graphical user interface. These include the \texttt{CPMView}, \texttt{CPMViewPanel}, and the \texttt{CPMControlPanel} classes. The \texttt{CPMView} manages the operations related to the window frame. The \texttt{CPMViewPanel} visualises the model by drawing individual domains or cells on the screen with different colours as the simulation progresses. The \texttt{CPMControlPanel} handles the parameters supplied by the user for running the model.  A third group of classes is responsible for the I/O processes. 

\begin{figure}[h]
\includegraphics[width=\textwidth]{figure/ClassDiagram.pdf}
\caption{A UML class diagram of the computer program developed which implements the CPM. Notice the program is divided into three groups of classes: the model classes, the visualisation classes, and the I/O classes. In addition, notice the implementation of the observer design pattern between the view and the model classes as a mean to reduce their coupling. The arrows represent the association and navigability between classes: the class at the tail of the arrow has a pointer to the class at the arrow head.}
\label{fig:ClassDiagram}
\end{figure}

The dependency between the model and visualisation classes is kept at a minimum to reduce the potential impact or changes required when one of them is altered. This is achieved by means of the observer design pattern. The model class extends a native java class Observable, while the visualisation classes implements the Observer interface, which requires the class to implement a method update() that is called when the Observable has changed.

To access the source code
For more details about the responsibilities and functionalities of each class, please see the Javadocs included in the source code. 

\subsubsection{Verification of the Implemented Model}
It is important to verify that the program implements the model correctly before using it to perform experiments. This was achieved via two approaches: Firstly, a set of unit test cases were developed to test individual methods of the model to ensure the calculations are correct. The specific test cases are included in the source code package (see appendix ?? for more information about accessing the source code). Secondly, the program was tested against results obtained results from other published papers that investigated the CPM. In particular, simulation data produced by the model are compared to those obtained by Szab\'o and coworkers in their study of the collective streaming behaviours of cells. By using the same set of parameters (i.e. a $200 \times 200$ lattice with 1000 cells and with $\alpha = 2$, $\lambda = 1.0$ , $P = 0.0$) and repeating the experiments with 1000 trials, the root-mean-square displacement (which is explained below in section ??) for an individual cell after 600 MCS agree with that reported by Szabo et al within the error of the measurement. Hence, there is a very high degree of confidence that the written program has implemented the model accurately.


\subsection{Experimental Methods}
\subsubsection{Statistical Measures of Glassy Dynamics}
To determine whether varying the parameters of the CPM will induce a jamming transition, suitable quantities are measured to determine whether the system is in the glassy regime. The two common indicators for glassy dynamics are the mean square displacement and the non-Gaussian parameter.

\paragraph{Mean Square Displacmeent}
The mean square displacement (MSD) of the system at a particular time $t$ is given by:
\begin{equation}
\langle{\bm{r}^2(t)\rangle} = \langle{\left(\bm{r}_i(t) - \bm{r}_i(t_0)\right)^2\rangle},
\end{equation}
where $\bm{r}_i (t)$ is the centre of mass of the $i$th cell at the time $t$, $t_0$ is a reference time frame, and $\langle...\rangle$ denotes an average over all cells of the system. The MSD measures the variance of the displacement of all cells from their original position at $t_0$. In a pure fluid, particles undergo Brownian motion due to the random collision with others. The particles' trajectories would be diffusive. According to Einstein's theory on Brownian dynamics, the MSD is expected to vary linearly with time and is given by the relation:
\begin{eqnarray}
\langle{\bm{r}^2(t)\rangle} = 2dDt, 
\end{eqnarray}
where $D$ is the diffusion coefficient and $d$ is the dimension of the system, which is 2 in this case. For a glassy system, the crowding effect alters the characteristic of the MSD. At short timescales, the dynamics of the cells are stuck due to the caging effect as mentioned above. This results in sub-diffusive behaviours, and the MSD scale with a lower exponent with respect to $t$. At large timescales, individual cells would have experienced many neighbour exchange events. As a whole, the dynamics of the cell return to the diffusive regime, and the MSD becomes linear with respect to $t$. These result in the classic plateau region when plotting the MSD curve against time in logarithmic scale as shown in figure \ref{fig:MSDExpectedCurve}. 
\begin{figure}[h]
\centering
\includegraphics[width=0.5\textwidth]{figure/MSDExpectedCurve.pdf}
\caption{The typical MSD versus time graph on a log-log plot. Notice the plateau region due to the sub-diffusive movement of the cells because of the caging effect. Diagram take from: \cite{hofling2013}.}
\label{fig:MSDExpectedCurve}
\end{figure}

\paragraph{Non-Gaussian Parameter}
Another widely used indicator for glassy dynamics is the non-Gaussian parameter. It measures the deviation from a Gaussian-like behaviour in the cells' trajectories. Mathematically, it is dimensionless ratio of the kurtosis of the particle's displacement distribution to its variance. In two dimension, this parameter is expressed as:
\begin{equation}
\alpha_2 = \frac{1}{2}\,\frac{\langle{\bm{r}^4(t)\rangle}}{\langle{\bm{r}^2(t)\rangle}^2} - 1,
\end{equation}
where $\langle{\bm{r}^4(t)\rangle}$ the fourth moment of . 

 For particles with diffusive behaviours, their displacement over time would have a Gaussian

\subsubsection{Order Parameters to Characterise Transition}
\paragraph{Diffusivity}
In order to quantify the existence of a phase transition, relevant order parameters are de which exhibit non-analytic behaviours at the critical point of the transition. As studied by Bi and coworkers, a dynamical order parameter to characterise the jamming transition is the self-diffusivity  of the cells. In a two dimensions, this is related to the mean square displacement by:

If the system is deep in the jammed and rigid-like regime, the cell dynamics would remain frozen for a long period of time. 

\paragraph{Diffusion Exponent}

\begin{equation}
\langle\bm{r}^2(t)\rangle = 4Dt^{\gamma}
\end{equation}

\subsubsection{Experiments Conducted}
The two parameters investigated are $\alpha$ and $P$, which, as discussed above, represent the strength of the interfacial effect between cells and the magnitude of a cell?s motility, respectively. 

All simulations were run on a 200 x 200 lattice with 1000 cells.  was set to 1.0 and  was set to 0.1. These parameters were chosen as they produce a physical representation of cells in tissues, and they were also used in other CPM simulations \cite{szabo2010}. Notice the value for  is irrelevant here since the focus is on confluent tissues, where there is no empty space between cells and hence no free boundaries. As mentioned above, periodic boundary conditions were assumed to eliminate the effects of external boundaries on the cell dynamics. 

For practical reasons, the same initial condition is used for all simulation. This should not result in correlation between measurements obtained in different trials, as the spin-update process is randomised and a different seed is used in each trial for the random number generator. The initial condition was prepared such that the system begins in a steady-state condition. This ensures the measurements are not influenced by transient behaviours. This was done by first initialising the cells as squares with equal area. They system was then allowed to run for 200000 MCS to reach steady-state with . This was chosen to  .  

%
%
%%This section should contain the details of the method employed. 
%%As in the previous sections standard techniques should not be written
%%out in detail. For example if you use an oscilloscope to take a
%%measurement, the theory of the CRO tube\footnote{Don't laugh, I have actually
%%seen this.} is {\bf not relevant}. In computational projects this
%%section should be used to explain the algorithms used and the layout of
%%the computational code. A copy of the acutal code must be
%%given in the appendices. Long detailed sections of theory, data tables
%%and details of computational code used in data analysis only should not
%%appear in this section, but should/may be included in the appendices.
%%
%%This section should emphasise the philosophy of the approach used
%%and detail novel techniques. However
%%please note: this section in {\bf not} a blow-by-blow account of what
%%you did throughout the project, and in particular it should {\bf not} 
%%contain large detailed sections about things you tried and found to be
%%completely wrong. Remember you are writing a technical report, and
%%not a diary. If however you find that a technique that was expected to
%%work failed, that is a valid result and should be included.
%%
%%Here logical structure is particularly important, and you may find that
%%to maintain good structure you may have to present the experiments
%%in a different order from the one in which you carried them out.
%

\pagebreak
\section{Results}

\subsection{Effects of Varying Cellular Interaction ($\alpha$)}
\begin{figure}[h]
\centering
\includegraphics[width=0.8\textwidth]{figure/R2_Alpha.pdf}
\caption{The mean square displacement of the cells versus time on a log-log plot for $\alpha$ range from 0.5 to 10.0 with increment by 0.5. As $\alpha$ increases, the dynamics of the cells are slowed down.}
\label{fig:r2alpha}
\end{figure}


\begin{figure}[h]
\centering
\begin{subfigure}[h]{0.49\textwidth}
\includegraphics[width=\textwidth]{figure/CM_Alpha_1.pdf}
\caption*{$\alpha = 1.0$}
\end{subfigure}
\begin{subfigure}[h]{0.49\textwidth}
\includegraphics[width=\textwidth]{figure/CM_Alpha_2.pdf}
\caption*{$\alpha = 2.0$}
\end{subfigure}
\begin{subfigure}[h]{0.49\textwidth}
\includegraphics[width=\textwidth]{figure/CM_Alpha_5.pdf}
\caption*{$\alpha = 5.0$}
\end{subfigure}
\begin{subfigure}[h]{0.49\textwidth}
\includegraphics[width=\textwidth]{figure/CM_Alpha_10.pdf}
\caption*{$\alpha = 10.0$}
\end{subfigure}
\caption{}
\end{figure}
\FloatBarrier

\begin{figure}[h]
\centering
\includegraphics[width=0.9\textwidth]{figure/DiffusivityExp_Alpha.pdf}
\caption{}
\end{figure}
\FloatBarrier

\subsection{Effects of Inducing Cell Motility $P$}

\begin{figure}[h]
\centering
\includegraphics[width=0.7\textwidth]{figure/Alpha_P_PhaseDiagram.pdf}
\caption{}
\end{figure}
\FloatBarrier

%
%This section should detail the obtained results in a clear,
%easy-to-follow manner. Remember long tables of numbers are just as boring to
%read as they are to type-in. Use graphs to present your results where
%-ever practicable. When quoting results or measurements
%{\bf DO NOT FORGET ABOUT ERRORS}. Remember there are two basic types
%of errors, these being random and systematic, which you must consider.
%Remember also the difference between an error and a mistake, computer
%program bugs are mistakes.
%
% 
%Again be selective in what you include. Half a dozen
%tables that contain totally wrong data you collected while you forgot
%to switch on the power supply are {\bf not relevant} and will frequently
%mask the correct results. 
%%
%%                       Here is how to inserted a centered
%%                       postscript file, this one is actually
%%                       out of Maple, but it will work for other
%%                       figures out of Xfig, Idraw and Xgraph
%%
%\begin{figure}[htb]     %Insert a figure as soon as possible
%        \begin{center}
%                \leavevmode             % Warn Latex a figure is comming
%                \epsfxsize=90mm         % Horizontal size YOUR want
%                                        % figure to be
%                \epsffile{otf.eps}
%\end{center}
%\caption{This is an inserted Postscript file}
%\end{figure}
%
%This section must contain a discussion of the results. This should
%include a discussion of the experimental and/or numerical errors, and a
%comparison with the predictions of the background and theory underlying
%the techniques used. This section should highlight particular strengths
%and/or weaknesses of the methods used.


\section{Discussion}
% 
\section{Conclusion}
%This section should summarise the results obtained, detail
%conclusions reached, suggest future work, and changes that you would make if you repeated the
%experiment. This section should in general be short, 100 to 150 words
%being typical for most projects.
%\par\noindent
%If you have opted to have multiple {\bf Theory, Method, Results}
%sections, draw all the results together in a {\bf single} conclusion.
%\section{References}
%
%Don't forget this section. Detail the relevant references which
%should be cited at the correct place in the text of the report. There
%are no fixed rules as to how many references are {\it needed}. Generally
%the longer the project, and the more background reading you had to do,
%the more references will be required. 
%
%When you cite a reference you must give sufficient information. For
%example, for a journal article give, {\it Author}, {\it Title of
%article},
%{\it Journal Name}, {\it Volumn}, {\it Page}, and {\it Year}, 
%while for a book give, {\it Author}, {\it Title},
%{\it (Editor if there is one)}, {\it Publisher}, and {\it Year}.        

%\bibliographystyle{unsrt}   
%\bibliography{bibliography}
\begin{thebibliography}{99}
	\bibitem{bi2015motility}
		D. Bi \etal
		Motility-driven glass and jamming transitions in biological tissues.
		\emph{arXiv}
		(2015)
	
	\bibitem{sanz2010}
		E. Sanz, D. Marenduzzo.
		Dynamic Monte Carlo versus Brownian dynamics: A comparison for self-diffusion and crystallization in colloidal fluids.
		\emph{J. Chem. Phys.}
		\textbf{132}, 194102 (2010)
	
	\bibitem{szabo2010}
		A. Szabo \etal
		Collective cell motion in endothelial monolayers.
		\emph{Phys. Biol.}
		\textbf{7} 046007 (2010)
		
	\bibitem{schoetz2013}
		E.-M. Schoetz \etal
		Glassy dynamics in three-dimensional embryonic tissues.
		\emph{J. R. Soc. Interface}
		\textbf{10} 20130726 (2013)
	
	\bibitem{angelini2010}
		T. E. Angelini \etal
		Glass-like dynamics of collective cell migration.
		\emph{Proc. Natl. Acad. Sci. U.S.A}
		\textbf{108} 4714 (2011)
	
	\bibitem{hecke2010}
		M. Hecke
		Jamming of soft particles: geometry, mechanics, scaling and isostaticity.
		\emph{J. Phys.: Condens. Matter}
		\textbf{22} 033101 (2010)
	
	\bibitem{liu2010}
		A. J. Liu, S. R. Nagel
		The jamming transition and the marginally jammed solid.
		\emph{Annu. Rev. Condens. Matter Phys.}
		\textbf{1} 347 (2010)
		
	\bibitem{graner1992}
		F. Graner, J. A. Glazier
		Simulation of biological cell sorting using a two-dimensional extended Potts model.
		\emph{Phys. Rev. Lett.}
		\textbf{69} 2013 (1992)
	
	\bibitem{bi2015density}
		D. Bi \etal
		A density-independent rigidity transition in biological tissues.
		\emph{Nat. Phys.}
		\textbf{11} 1074 (2015)
		
	\bibitem{kabla2012}
		A. J. Kabla
		Collective cell migration: leadership, invasion and segregation.
		\emph{J. R. Soc. Interface}
		\textbf{9} 3268 (2012)
	
	\bibitem{nagai2001}
		T. Nagai, H. Honda
		A dynamic cell model for the formation of epithelial tissues.
		\emph{Philos. Mag. Part B}
		\textbf{81} 699 (2001)
	
	\bibitem{friedl2009}
		P. Friedl, D. Gilmour
		Collective cell migration in morphogenesis, regeneration and cancer.
		\emph{Nat. Rev. Mol. Cell Biol.}
		\textbf{10} 445 (2009)
	
	\bibitem{liu1998}
		A. J. Liu, S. R. Nagel
		Nonlinear dynamics: Jamming is not just cool any more.
		\emph{Nature}
		\textbf{396} 21 (1998)
		
	\bibitem{hofling2013}
		F. H\"ofling, T. Franosch
		Anomalous transport in the crowded world of biological cells.
		\emph{Rep. Prog. Phys.}
		\textbf{76} 046602 (2013)

\end{thebibliography}

\appendix
\section{Appendices}

%Material that is useful background to the report, but is not essential,
%or whose inclusion within the report  would detract from its
%structure and readablity, should be included in appendices. Typical
%material could be diagrams of electronic circuits built, specialist
%data tables used to analyse results, details of computer programs
%written for analysis and display of results, photographic plates,
%and, for computational projects, a copy of all written code.
%
%Again be selective. The appendix is {\bf not} an excuse for you to add every
%last detail and piece of data, but should be used to assist the reader
%of the report by supplying additional material. Not all reports require
%appendices and if the report is complete without this additional
%material leave it out.

\subsection{Simulation Code}

\end{document}
