%
%                       This is a LaTeX 2e version of the
%                       laboratory project template file.
\documentclass[a4paper,12pt]{article}
\usepackage{fullpage,epsf}
\usepackage{amsmath}
\usepackage{amsfonts}
\usepackage{amssymb}
\usepackage{amstext}
\usepackage{bm}
\usepackage{braket}
\usepackage{array}
\usepackage{graphicx}
\usepackage{tabularx}
\usepackage{url}
\usepackage{verbatim}
\usepackage{listings}
\usepackage{color}
\usepackage{courier}
\usepackage{epstopdf}
\usepackage{placeins}
\usepackage[font=footnotesize]{caption}
\epstopdfsetup{update}
\usepackage[]{cite}


%%%%%%%%%%%%%%%%%%%%%%%%%%%%%%%%%%%%%%%%%%%%%%%%%%%%%
\renewcommand{\vec}[1]{\mathbf{#1}}
\newcommand{\abs}[1]{\left|#1\right|}
\newcommand{\inc}{\Delta}
\newcommand{\etal}{\textit{et al. }}
%%%%%%%%%%%%%%%%%%%%%%%%%%%%%%%%%%%%%%%%%%%%%%%%%%%%%

%
%                       This section generates a title page
%                       Edit only the sections indicated to put
%                       in the project title, your name, supervisor,
%                       project length in weeks and submission date
%
\begin{document}
\pagestyle{empty}                       % No numbers of title page                      
\epsfxsize=40mm                         % Size of crest
\begin{minipage}[b]{110mm}
        {\Huge\bf School of Physics\\ and Astronomy
        \vspace*{17mm}}
\end{minipage}
\hfill
\begin{minipage}[t]{40mm}               
        \makebox[40mm]{
        \includegraphics[width=4cm]{crest.jpg}}
\end{minipage}
\par\noindent                                           % Centre Title, and name
\vspace*{2cm}
\begin{center}
        \Large\bf \Large\bf Senior Honours Project\\
        \Large\bf Physics 4\\[10pt]                     % Change to MP/CP/Astro
        \LARGE\bf Investigation of the \\Jamming Transition in Cellular Potts Model       % Change to suit
\end{center}
\vspace*{0.5cm}
\begin{center}
        \bf Michael Chiang\\                           % Repace with your name
        25 March 2015                                    % Submission Date
\end{center}
\vspace*{5mm}
%
%                       Insert your abstract HERE
%                       
\begin{abstract}
        The abstract is a short, concise explanation of the project
        covering the aims, outlines of techniques used and a short
        summary of the results. It should contain enough information to
        make the aims and success of the project clear, but contain no details.
        A typical abstract should be between 50 and 100 words.
\end{abstract}

\vspace*{1cm}

\subsubsection*{Declaration}

\begin{quotation}
        I declare that this project and report is my own work.
\end{quotation}

\vspace*{2cm}
Signature:\hspace*{8cm}Date: 25 March 2015

\vfill
{\bf Supervisor:} Prof. D. Marenduzzo                 % Change to suit
\hfill
10 Weeks                                         % Change to suit
\newpage
%
%                       End of Title Page
\pagestyle{plain}                               % Page numbers at bottom
\setcounter{page}{1}                            % Set page number to 1
\tableofcontents                                % Makes Table of Contents

\break
\section{Introduction}
Cell monolayers are widely studied biophysical systems as they are a simple model of a tissue. They provide an economic approach to study complex biological processes that take place in multi-cellular organisms. A particular research area where cell monolayers have been used as a model is in the study of dense biological tissues. These tissues, where cells are closely packed against each other with minimal spacing in between, are often found in important physiological and pathological processes such as embryonic development, wound healing, and cancer metastasis [Friedl2009].  

Recent in vitro experiments have demonstrated that these dense tissues exhibit both fluid and glass-like behaviours [Angelini, Schoetz]. This is reminiscent of the jamming transition observed in condensed matter, which describes the transition of matter with disordered structure from a fluid (unjammed) to rigid (jammed) state. This transition is widely studied in non-living systems such as colloidal dispersions, granular matter, foams, and emulsions. 

Computer simulations have been conducted to further understand the mechanics behind the fluid and rigid behaviours of tissues. To this end, multiple models have been developed to describe the dynamics of cells in biological tissues. A particular successful model is the vertex model, where cells are represented as a polygonal tiling of space and their dynamics are governed by a set of force equations on the vertices of the polygons [Nagai2001]. A recent computational study indicates that model exhibits a jamming transition in accordance to the empirical observations, with the cell shape being a key parameter in determining the rigidity of the tissue [Bi2015].

Another widely used model in simulating cell dynamics in tissues is the cellular Potts model. This model represents cells as domains on a lattice and their motions are governed by minimising an energy function that characterises the cellular interactions. There, however, has not been simulations conducted to verify whether the model has a fluid to rigid transition. With this in mind, the aim of this project is to implement the cellular Potts model and to investigate whether varying the model parameters would reproduce the jamming transition observed empirically in cell monolayers. 

The remain sections of this paper is organised as follows: Section 2 provides a brief introduction to the cellular Potts model, the concept of jamming, and its connection to the dynamics of dense biological tissues. Section 3 outlines the implementation of the model and the experimental methods for investigating whether CPM can induce the jamming transition. Section 4 and 5 present and discuss the results obtained from the simulations. 


%The introduction section of the report should introduce the project in
%more detail than in the abstract. In particular it should present the
%motivation, the aims, outline of techniques used, and the scope of the project. 
%It should also contain references to similar work in the
%same field to put your work in the correct context.
%
%As a general rule, people reading the abstract and introduction alone
%should have a good idea of the material in the project, the techniques
%employed and the results obtained. A typical introduction should be
%about 1 page, (300-450 words). 


\section{Background and Theory}

%This section should cover the theory of the material in the project
%in sufficient detail to make the following work understandable to the
%average physicist. It should not contain large sections of standard
%bookwork, but should contain references to this material. The exact
%contents of this section will depend on the project being undertaken.
%
%This section should contain only the
%relevant theory. In particular a life history of the inventor of the
%technique to be used is
%totally irrelevant\footnote{I have seen a report that contained three pages
%on the life of Gabor, and it was not very interesting.}. Here use common sense
%and the general rule, ``If in doubt: leave it out'', however
%include information that you judge would be useful to one of your
%peers if they wehe to repeat the project. If you are
%undertaking a 12 week project and it includes a literature search, put the
%result of the search here. As a rough guide this section should be
%about 3-4 pages for a 6 week project, longer for longer projects.
%
%Note that if the project consists of a series of short experiments
%each of which requires a different theory and method, it may be appropriate
%to have one {\bf Theory, Method, Results} section for each
%experiment.

\subsection{Cellular Potts Model}
Developed by Glazier and Graner\cite{graner1992}, the cellular Potts model (CPM)\footnote{The CPM is also known as the GGH model, naming after Glazier and Graner who first proposed it and Hogeweg who extended it for various biological modelling applications.} is a stochastic model widely used in simulating the motion of closely packed cells in a monolayer. The model is an extension of the Potts model used in statistical physics and takes into account of the physical characteristics and dynamics of biological cells. The model has successfully explained collective behaviours such as cell sorting \cite{graner1992} and streaming\cite{szabo2010}.

In the CPM, a cell monolayer is represented on a two dimensional lattice. Each lattice site is assigned with a non-negative integer spin $\sigma$ known as the cell index, where $\sigma \in [0,N]$ and $N$ is the total number of cells. Two lattice sites which share a side or a corner are considered as neighbours. A biological cell is denoted by a set of connected sites with the same spin. $\sigma = 0$ is reserved for enumerating sites that are not part of any cell. Throughout this project, periodic boundary condition is applied such that the boundaries have no influence on the cells dynamics.

\begin{figure}[h]
\centering
\includegraphics[width=7.0cm]{figure/CPMDiagram.png}
\caption{A pictorial representation of the cellular Potts model. A cell is represented by a domain in the lattice, which is a set of interconnecting sites with the same spin value. Diagram taken from Glazier \etal, Single Cell Based Models in Biology and Medicine [cite].}
\label{fig:CPMDiagram}
\end{figure}

The dynamics of the cells are evolved using a modified version of the Metropolis algorithm: In the simulation, a series of elementary steps are taken with the goal of minimising an effective Hamiltonian H that characterises the properties of cells and their interactions. Each step is an attempt to change the spin of a random lattice site $\bm{a}$ to that of a randomly chosen neighbour site $\bm{b}$, where $\sigma(\bm{a}) \neq \sigma(\bm{b})$. The probability of accepting a spin change, $p\left(\sigma(\bm{a}) \rightarrow \sigma(\bm{b})\right)$, is determined by the associated change in energy ($\inc H$): if $\inc H \leq 0$, the spin change is always accepted; if $\inc H > 0$, the spin change is accepted with a probability of $\exp⁡(-\inc H / kT)$, where $k$ is the Boltzmann constant and $T$ is the effective temperature. These relations can be conveniently summarised by the following expression:
\begin{eqnarray}
\ln p\left(\sigma(\bm{a}) \rightarrow \sigma(\bm{b})\right) = \min\left[0,-\frac{\inc H}{kT}\right].
\end{eqnarray}
The exact form of the Hamiltonian will be discussed below. Since changing $k$ and $T$ is equivalent to multiplying the energy function by an irrelevant constant, it is conventional to set $k$ and $T$ to $1$ and express the energy in units of $kT$. As updating the entire lattice takes more steps with a larger system, the number of elementary steps taken is not a proper measure of time. It is conventional to define a unit time as $L^2$ number of spin change attempts, where $L$ is the linear dimension of the lattice. This is known as one Monte-Carlo step (MCS). One also defines the acceptance rate as the ratio of the number of spin copies accepted to the number of attempts made. This is important for an accurate comparison between the time scales used in simulations and those used in experiments\cite{sanz2010}.

As mentioned above, the dynamics of the cells are governed by an effective Hamiltonian that describes the physical behaviours of cells in tissues. In particular, the CPM captures the following three aspects of cellular behaviours observed empirically:
\begin{enumerate}
\item Cells adhere with each other and have surface tension.
\item Cells are approximately the same size due to incompressibility and resistance to height fluctuations.
\item Cells are capable of generating a motile force and move along a polarised direction that changes over time.
\end{enumerate}
These aspects are encoded in the model by the following Hamiltonian:
\begin{eqnarray}
\label{eqn:hamiltonian}
H = \sum_{\langle{\bm{x}, \bm{x'}\rangle}} J\left(\sigma(\bm{x}), \sigma(\bm{x'})\right) + \lambda \sum_{i\,=\,1}^{N} \left(a_i - A_i\right)^2 - P \;\sum_{i\,=\,1}^{N} \bm{n}_i \cdot \bm{r}_i.
\end{eqnarray}
The first term accounts for the overall interfacial behaviours of cells. This includes cell-cell adhesion and the surface tension of each cell. The term is a sum over the interfacial energy for all pairs of neighbours in the lattice. $J$ describes the specific interfacial energy between two sites $\bm{x}$ and $\bm{x}'$. For simulating an ensemble of homogenous cells, $J$ is specified by the following values:
\begin{eqnarray}
\label{eqn:interfacial_energy}
J(\sigma,\sigma') = \left\{
	\begin {array}{ll}
		0 & \textrm{for $\sigma = \sigma'$}\\
		\alpha & \textrm{for $\sigma, \sigma' > 0$ and $\sigma \neq \sigma'$}\\
		\beta & \textrm{for $\sigma$ or $\sigma' = 0$},
	\end{array}
\right.
\end{eqnarray}
where $\alpha$ and $\beta$ are parameters for modelling the interfacial energy between two cells and the free boundary energy, respectively. Previous studies have found that the magnitudes of these parameters have the effect on the cell shape and the flexibility of the cell boundaries: small magnitudes produce long and rough boundaries with more active cell dynamics, while large magnitudes result in straight boundaries with less motions in the cells\cite{szabo2010}.

The second term characterises the incompressibility of cells and their resistance to height fluctuations. This is modelled as an elastic constraint in the Hamiltonian. $a_i$ and $A_i$ are the current and target area of the $i$th cell. $\lambda$ sets the strength of this constraint and governs the magnitude of the area fluctuation of each cell. 

The third term is proposed by Szabo \etal in their study of the streaming behaviours in cell monolayers\cite{szabo2010}. It introduces a bias in the evolution of the system to favour spin changes that result in the cells moving in their desired direction. In this term, $\bm{n}_i$ is a polarity vector that describes the preferred direction of the cell,  $\bm{r}_i$ is the centre of mass of the cell, and $P$ is a free parameter associated with the strength of the motility, which is related to the cell's speed (taken to be homogenous). The specific mechanisms of how a cell generates a polarized motion and how the polarity changes over time remain poorly understood and are subject to intense investigations [cite]. A simple model, as suggested by Bi \etal and is implemented in this work, is to assume that the polarity vector undergoes rotational diffusion over time. If one defines the polarity angle of the $i$th cell ($\theta_i$) as the angle between the cell's polarity vector ($\bm{n}_i$) and a reference axis (taken to be the $x$-axis herein), this diffusion process is governed by the following equations:
\begin{eqnarray}
\partial_t\theta_i(t) & = & \eta_i(t)\\
\langle{\eta_i(t)\eta_j(t')\rangle} & = & 2D_r\delta(t-t')\delta_{ij}
\end{eqnarray}
where $\eta_i$ is a noise function with zero mean and a variance of $2D_r$ with $D_r$ specifying the magnitude of the angular fluctuations. On a computer, this can be represented by:
\begin{eqnarray}
\label{eqn:computerRotateDiff}
\theta_i(t+\Delta t) = \theta_i(t) + \sqrt{2D_r\Delta t}\,\tilde\eta
\end{eqnarray}
where $\tilde\eta$ is a random number generated in the range $[-1,1)$. The last term in equation \ref{eqn:computerRotateDiff} reproduces the statistical properties of the noise function. 

\subsection{Jamming Transitions in Biological Tissues}
In condensed matter, jamming describes the transition from a flowing to rigid state of a material with its structure remaining disordered in the process. This occurs in various form of soft matter such as emulsions, foams, granular materials, and colloidal dispersions\cite{hecke2010}. For example, colloidal particles can jam and form a glassy structure when the packing fraction is increased beyond a critical point, while foams can lose its ability to flow when the applied stress is lowered. These materials, however, can unjam and recover their fluid-like properties when the suitable control parameter is varied. The conditions for the transition between the fluid and rigid structure is commonly summarised by the ``jamming phase diagram''\cite{liu2010}.

\begin{figure}[h]
\centering
\includegraphics[width=9.0cm]{figure/JammingPhaseDiagram.jpeg}
\caption{The jamming phase diagram proposed by Liu and Nigel.}
\label{fig:JammingPhaseDiagram}
\end{figure}

Recent studies have indicated that jamming transitions also occur in dense biological tissues. Experiments have shown that these tissues exhibit typical glass-like behaviours such as caging, dynamical heterogeneity, and viscoelasticity \cite{schoetz2013,angelini2011}. In addition, these tissues participate in complex biological processes such as wound healing, embryonic development, and cancer metastasis. During these processes, the tissues can undergo the Epithelial-Mesenchymal Transition (EMT), where rigid-like epithelial cells become fluid-like mesenchymal phenotypes (i.e. loosely associated cells), or the inverse process, the Mesenchymal-Epithelial-Transition (MET). These transitions are very similar to jamming of soft materials. 

There have been suggestions that the collective motion of cells plays a role in explaining this transition process. Researchers have found that the transition point is closely related to the cell shapes\cite{bi-density2015}. The shapes can be affected by factors such as cell-cell adhesion strength and active cortical tension, which is an effective surface tension.

Numerical modelling and computer simulations have been performed to describe the collective motion of cells with various degrees of success. An example is the self-propelled particle (SPP) model, where cells are represented as disks or spheres like colloidal particles. Another example is the vertex model, where cells are represented by a set of interconnecting vertices and their dynamics are governed by a set of force equations. A further example, which is also the model implemented in this project, is the Cellular Potts Model.


\subsection{Motivation of the Study}
Recently, Bi and co-workers demonstrated that jamming in confluent tissues can be described using another model known as the Self-Propelled Voronoi (SPV) model and investigated the effects of cell motility on the transition condition\cite{bimotility-driven2015}. This model is a variation of the vertex model where cells are specified by the centres of enclosing vertices. The dynamics of the cells is governed by force equations on the centres which take into account of the intercellular effects such as cell-cell adhesion and cortical tension.

In that study, researchers found that the phase transition can be affected by several parameters such as the speed of a single motile cell, the persistent time, and the cell shape index. The persistent time is the characteristic time in which a cell would travel before changing its direction. The cell shape index is a dimensionless ratio between the perimeter of the cell, $P$, to the square root of its cross section area, $A$ (i.e. $p_0=P/\sqrt{A}$). In the limit of zero cell motility, the simulation results agree with previous studies that $p_0^*=3.81$ is critical value where phase transition occurs. Furthermore, when the cells are motile with rotational diffusion, the results show that $p_0^*$ reduces as the motility increases.  

Given the success of CPM in modelling various collective cell phenomena, it seems reasonable that the model should also be capable of describing the jamming transitions. In particular, both SPV and CPM take into account of intercellular effects such as intercellular adhesion and tensions. Although CPM does not allow one to explicitly alter the preferred perimeter of a cell and thus $p_0$, which is possible in SPV, it is likely that an equivalent effect can be achieved indirectly by changing the interfacial energy $\alpha$ between cells (see equation \ref{eqn:interfacial_energy}). This is because a lower $\alpha$ tends to give a rougher boundary, which would result in a higher $p_0$. In contrast, a higher $\alpha$ tends to give a smoother boundary with a lower $p_0$. 

This project investigates whether the CPM can describe the jamming transition. The main objective is to determine whether the variation on the model parameter $\alpha$, which models the surface tension and cell-cell adhesion between cells, has the ability to induce a fluid-to-glass transition. Another objective is to understand the effect of cell motility on the transition. 



\section{Methodology}
As discussed in the previous section, the project is divided into two section. The first part of the project focuses on implementing the cellular Potts model on the computer; the second part is devoted to the experimentation of the model parameters to investigate whether the model exhibits the jamming phase transition that is observed in cell monolayers.

\subsection{Implementation of the Cellular Potts Model}
\subsubsection{Programming Language and Software Used}
The program is written in Java as it is a robust, agnostic language. It also provides an easy-to-use framework for developing graphical user interfaces, which is needed in this project to visualise the cell motions. To facilitate the development process, the Eclipse integrated development environment (IDE) is used as it has a comprehensive code browsing and debugging toolkit. In addition, the version control system, git, is used to maintain the history of the development and facilitate the synchronisation of the code across multiple platforms. The ANT automated build system is also used in this project to allow rapid compilation of the code.

\subsubsection{Code Design and Structure}
The main structure and design of the code is described as follow: There are three main groups of classes in the program (see figure ??). Firstly, there are the model classes that handle the main computation as specified by the CPM. These include the CellPottsModel and the SpinModel class. CellPottsModel contains most of the logic of the model, including the Metropolis algorithm for updating spins and the evaluation of the various statistical quantities measured in the experiments. The SpinModel is an interface  . The second group is the visualisation classes which display the model to the user in a graphical user interface. These include the CPMView, CPMViewPanel, and the CPMControlPanel classes. The CPMView handles the operations related to the window frame. The CPMViewPanel visualises the model by . The CPMControlPanel handles the parameters supplied by the user for running the model.  A third group of classes is responsible for the I/O processes. 

\begin{figure}[h]
\includegraphics[width=\textwidth]{figure/ClassDiagram.pdf}
\caption{A UML class diagram of the computer program developed which implements the CPM. Notice the program is divided into three groups of classes: the model classes, the visualisation classes, and the I/O classes. Notice the implementation of the observer design pattern between the view and the model classes as a mean to reduce their coupling. The arrows represent the association and navigability between classes: the class at the tail of the arrow has a pointer to the class at the arrow head.}
\label{fig:ClassDiagram}
\end{figure}

The dependency between the model and visualisation classes are kept at a minimum to reduce the potential impact or changes required when one of them is updated. This is achieved by means of the observer design pattern. The model class extends a native java class Observable, while the visualisation classes implements the Observer interface, which requires the class to implement a method update() that is called when the Observable has changed. 

\subsubsection{Verification of the Implemented Model}
It is important to verify that the program implements the model correctly before using it to perform experiments. This was achieved via two approaches: firstly, a set of unit test cases were developed to test individual methods of the model to ensure the calculations are correct. The specific test cases are included in the source code package which is available on GitHub (see appendix ?? for more information about accessing the source code). Secondly, the program was tested against results obtained results from other published papers that investigated the CPM. In particular, simulation data produced by the model are compared to those obtained by Szabo and coworkers in their study of the collective streaming behaviours of cells. By using the same set of parameters (i.e. a 200 x 200 lattice with 1000 cells and with α=2, λ=1.0, P=0.0) and repeating the experiments with 1000 trials, the root-mean-square displacement (which is explained below in section ??) for an individual cell after 600 MCS agree with that reported by Szabo et al within the error of the measurement. Hence, there is a very high degree of confidence that the written program has implemented the model accurately.





\subsection{Experimental Methods}
\subsubsection{Statistical Measures of Glassy Dynamics}
To determine whether varying the parameters of the CPM will induce a jamming transition, suitable quantities are measured to determine whether the system is in the glassy regime. The two common indicators for glassy dynamics are the mean square displacement and the non-Gaussian parameter.

\paragraph{Mean Square Displacmeent}
The mean square displacement (MSD) of the system at a particular time $t$ is given by:
\begin{equation}
\langle{\bm{r}^2(t)\rangle} = \langle{\left(\bm{r}_i(t) - \bm{r}_i(t_0)\right)^2\rangle}
\end{equation}
where $\bm{r}_i (t)$ is the centre of mass of the ith cell at the time $t$, $t_0$ is a reference time frame, and $\langle...\rangle$ denotes an average over all cells of the system. The MSD measures the variance of the displacement of all cells from their original position at $t_0$. 

\paragraph{Non-Gaussian Parameter}

\begin{equation}
\alpha_2 = \frac{3}{5}\,\frac{\langle{\bm{R}^4(t)\rangle}}{\langle{\bm{R}^2(t)\rangle}} - 1
\end{equation}

\subsubsection{Order Parameters to Characterise Transition}
\paragraph{Diffusivity}

%
%\subsection{Code Structure}
%
%%This section should contain the details of the method employed. 
%%As in the previous sections standard techniques should not be written
%%out in detail. For example if you use an oscilloscope to take a
%%measurement, the theory of the CRO tube\footnote{Don't laugh, I have actually
%%seen this.} is {\bf not relevant}. In computational projects this
%%section should be used to explain the algorithms used and the layout of
%%the computational code. A copy of the acutal code must be
%%given in the appendices. Long detailed sections of theory, data tables
%%and details of computational code used in data analysis only should not
%%appear in this section, but should/may be included in the appendices.
%%
%%This section should emphasise the philosophy of the approach used
%%and detail novel techniques. However
%%please note: this section in {\bf not} a blow-by-blow account of what
%%you did throughout the project, and in particular it should {\bf not} 
%%contain large detailed sections about things you tried and found to be
%%completely wrong. Remember you are writing a technical report, and
%%not a diary. If however you find that a technique that was expected to
%%work failed, that is a valid result and should be included.
%%
%%Here logical structure is particularly important, and you may find that
%%to maintain good structure you may have to present the experiments
%%in a different order from the one in which you carried them out.
%


\section{Results}
%
%This section should detail the obtained results in a clear,
%easy-to-follow manner. Remember long tables of numbers are just as boring to
%read as they are to type-in. Use graphs to present your results where
%-ever practicable. When quoting results or measurements
%{\bf DO NOT FORGET ABOUT ERRORS}. Remember there are two basic types
%of errors, these being random and systematic, which you must consider.
%Remember also the difference between an error and a mistake, computer
%program bugs are mistakes.
%
% 
%Again be selective in what you include. Half a dozen
%tables that contain totally wrong data you collected while you forgot
%to switch on the power supply are {\bf not relevant} and will frequently
%mask the correct results. 
%%
%%                       Here is how to inserted a centered
%%                       postscript file, this one is actually
%%                       out of Maple, but it will work for other
%%                       figures out of Xfig, Idraw and Xgraph
%%
%\begin{figure}[htb]     %Insert a figure as soon as possible
%        \begin{center}
%                \leavevmode             % Warn Latex a figure is comming
%                \epsfxsize=90mm         % Horizontal size YOUR want
%                                        % figure to be
%                \epsffile{otf.eps}
%\end{center}
%\caption{This is an inserted Postscript file}
%\end{figure}
%
%This section must contain a discussion of the results. This should
%include a discussion of the experimental and/or numerical errors, and a
%comparison with the predictions of the background and theory underlying
%the techniques used. This section should highlight particular strengths
%and/or weaknesses of the methods used.


\section{Discussion}
% 
\section{Conclusion}
%This section should summarise the results obtained, detail
%conclusions reached, suggest future work, and changes that you would make if you repeated the
%experiment. This section should in general be short, 100 to 150 words
%being typical for most projects.
%\par\noindent
%If you have opted to have multiple {\bf Theory, Method, Results}
%sections, draw all the results together in a {\bf single} conclusion.
%\section{References}
%
%Don't forget this section. Detail the relevant references which
%should be cited at the correct place in the text of the report. There
%are no fixed rules as to how many references are {\it needed}. Generally
%the longer the project, and the more background reading you had to do,
%the more references will be required. 
%
%When you cite a reference you must give sufficient information. For
%example, for a journal article give, {\it Author}, {\it Title of
%article},
%{\it Journal Name}, {\it Volumn}, {\it Page}, and {\it Year}, 
%while for a book give, {\it Author}, {\it Title},
%{\it (Editor if there is one)}, {\it Publisher}, and {\it Year}.        

\bibliographystyle{unsrt}   
\bibliography{bibliography}


\appendix
\section{Appendices}

%Material that is useful background to the report, but is not essential,
%or whose inclusion within the report  would detract from its
%structure and readablity, should be included in appendices. Typical
%material could be diagrams of electronic circuits built, specialist
%data tables used to analyse results, details of computer programs
%written for analysis and display of results, photographic plates,
%and, for computational projects, a copy of all written code.
%
%Again be selective. The appendix is {\bf not} an excuse for you to add every
%last detail and piece of data, but should be used to assist the reader
%of the report by supplying additional material. Not all reports require
%appendices and if the report is complete without this additional
%material leave it out.

\subsection{Simulation Code}

\end{document}
