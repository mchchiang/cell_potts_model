%
%                       This is a LaTeX 2e version of the
%                       laboratory project template file.
\documentclass[a4paper,12pt]{article}
\usepackage{fullpage,epsf}
\usepackage{amsmath}
\usepackage{amsfonts}
\usepackage{amssymb}
\usepackage{amstext}
\usepackage{bm}
\usepackage{braket}
\usepackage{array}
\usepackage{graphicx}
\usepackage{tabularx}
\usepackage{url}
\usepackage{verbatim}
\usepackage{listings}
\usepackage{color}
\usepackage{courier}
\usepackage{epstopdf}
\usepackage{placeins}
\epstopdfsetup{update}
\usepackage[]{cite}


%%%%%%%%%%%%%%%%%%%%%%%%%%%%%%%%%%%%%%%%%%%%%%%%%%%%%
\renewcommand{\vec}[1]{\mathbf{#1}}
\newcommand{\abs}[1]{\left|#1\right|}
\newcommand{\inc}{\Delta}

%%%%%%%%%%%%%%%%%%%%%%%%%%%%%%%%%%%%%%%%%%%%%%%%%%%%%

%
%                       This section generates a title page
%                       Edit only the sections indicated to put
%                       in the project title, your name, supervisor,
%                       project length in weeks and submission date
%
\begin{document}
\pagestyle{empty}                       % No numbers of title page                      
\epsfxsize=40mm                         % Size of crest
\begin{minipage}[b]{110mm}
        {\Huge\bf School of Physics\\ and Astronomy
        \vspace*{17mm}}
\end{minipage}
\hfill
\begin{minipage}[t]{40mm}               
        \makebox[40mm]{
        \includegraphics[width=4cm]{crest.jpg}}
\end{minipage}
\par\noindent                                           % Centre Title, and name
\vspace*{2cm}
\begin{center}
        \Large\bf \Large\bf Senior Honours Project\\
        \Large\bf Physics 4\\[10pt]                     % Change to MP/CP/Astro
        \LARGE\bf Simulating the Jamming Transitions of a Cellular Monolayer with the Cellular Potts Model        % Change to suit
\end{center}
\vspace*{0.5cm}
\begin{center}
        \bf Michael Chiang\\                           % Repace with your name
        25 March 2015                                    % Submission Date
\end{center}
\vspace*{5mm}
%
%                       Insert your abstract HERE
%                       
\begin{abstract}
        The abstract is a short, concise explanation of the project
        covering the aims, outlines of techniques used and a short
        summary of the results. It should contain enough information to
        make the aims and success of the project clear, but contain no details.
        A typical abstract should be between 50 and 100 words.
\end{abstract}

\vspace*{1cm}

\subsubsection*{Declaration}

\begin{quotation}
        I declare that this project and report is my own work.
\end{quotation}

\vspace*{2cm}
Signature:\hspace*{8cm}Date: 25 March 2015

\vfill
{\bf Supervisor:} Prof. D. Marenduzzo                 % Change to suit
\hfill
10 Weeks                                         % Change to suit
\newpage
%
%                       End of Title Page
\pagestyle{plain}                               % Page numbers at bottom
\setcounter{page}{1}                            % Set page number to 1
\tableofcontents                                % Makes Table of Contents

\break
\section{Introduction}
The study of cellular monolayers has been an active field in biological physics. Cellular monolayers can be used as a simple model of a tissue. 

%The introduction section of the report should introduce the project in
%more detail than in the abstract. In particular it should present the
%motivation, the aims, outline of techniques used, and the scope of the project. 
%It should also contain references to similar work in the
%same field to put your work in the correct context.
%
%As a general rule, people reading the abstract and introduction alone
%should have a good idea of the material in the project, the techniques
%employed and the results obtained. A typical introduction should be
%about 1 page, (300-450 words). 

\pagebreak
\section{Background and Theory}

%This section should cover the theory of the material in the project
%in sufficient detail to make the following work understandable to the
%average physicist. It should not contain large sections of standard
%bookwork, but should contain references to this material. The exact
%contents of this section will depend on the project being undertaken.
%
%This section should contain only the
%relevant theory. In particular a life history of the inventor of the
%technique to be used is
%totally irrelevant\footnote{I have seen a report that contained three pages
%on the life of Gabor, and it was not very interesting.}. Here use common sense
%and the general rule, ``If in doubt: leave it out'', however
%include information that you judge would be useful to one of your
%peers if they wehe to repeat the project. If you are
%undertaking a 12 week project and it includes a literature search, put the
%result of the search here. As a rough guide this section should be
%about 3-4 pages for a 6 week project, longer for longer projects.
%
%Note that if the project consists of a series of short experiments
%each of which requires a different theory and method, it may be appropriate
%to have one {\bf Theory, Method, Results} section for each
%experiment.

\subsection{Jamming Transitions in Biology}

The jamming\cite{bimotility-driven2015}

\subsection{Cellular Potts Model}


The Cellular Potts Model (CPM) is a stochastic model widely used in simulating the motion of closely packed cells. The model is an extension of the Potts model that takes into account of the physical characteristics and dynamics of biological cells. The model has successfully explained cellular processes such as cell sorting [Glazier], collective streaming [Szabo], and other behaviours.

In the CPM, biological cells are represented on a two dimensional lattice. Each lattice site is assigned with a non-negative integer spin $\sigma$, where $\sigma \in [0,N]$ and $N$ is the total number of cells. Two lattice sites which share a side or a corner are considered as neighbours. A biological cell is simply represented by a set of connected sites with the same spin. $\sigma = 0$ is reserved for enumerating sites that are not part of any cell.


The dynamics of the cells are using a modified version of the Metropolis algorithm. In the simulation, a series of elementary steps are taken. Each step is an attempt to change the spin of a randomly chosen lattice site, denoted as $\sigma_{\textrm{target}}$, to the spin of one of its neighbours, denoted as $\sigma_{\textrm{trial}}$. The probability of accepting the spin change, $p\left(\sigma(\bm{a}) \rightarrow \sigma(\bm{b})\right)$, is given by the relation:
\begin{eqnarray}
\label{eqn:prob}
\ln p (\bm{a} \rightarrow \bm{b}) = \min \left[0, \frac{-\inc H (\bm{a} \rightarrow \bm{b}) + w (\bm{a} \rightarrow \bm{b})}{kT} \right],
\end{eqnarray}
where $w$ is a function to take into account of active cell motion, $H$ is an effective Hamiltonian which the system tries to minimise,  $\inc H (\bm{a} \rightarrow \bm{b})$ represents its change in the considered step, $T$ is the effective temperature, and $k$ is the Boltzmann constant. The functional form of $H$ and $w$ would be discussed further below. Since changing $k$ and $T$ is equivalent to multiplying both the $H$ and $w$ function by a constant, it is conventional to set $k$ and $T$ to 1 and rescale parameters in $H$ and $w$ accordingly. As the typical time for all lattice sites to have experienced an attempt depends on the lattice size, it is it is conventional to define a unit time as $N$ number of spin-copy attempts. This is known as one Monte-Carlo step (MCS). One also defines the acceptance rate as the ratio of the number of spin copies accepted to the number of attempts made. This is important for an accurate comparison between the time scales in the simulations and those used in experiments \cite{sanz2010}.
 
Equation \ref{eqn:prob} shows that the CPM dynamics is governed by an effective Hamiltonian that models the physical characteristics of a biological cell. This includes the intercellular adhesion and a constraint on the 
\begin{eqnarray}
\label{eqn:hamiltonian}
H = \sum_{\langle{\bm{x}, \bm{x'}\rangle}} J\left(\sigma(\bm{x}), \sigma(\bm{x'})\right) + \lambda \sum_{q\,=\,1}^{N} \left(a_q - A_q\right)^2,
\end{eqnarray}
The first term in equation \ref{eqn:hamiltonian} describes the cell-cell adhesion strength. It is a sum over all pairs of neighbours in the lattice, and J describes the adhesion strength between two sites $\bm{x}$ and $\bm{x'}$. For simulating an ensemble of homogenous cells, $J$ is specified by the following values:
\begin{eqnarray}
J(\sigma,\sigma') = \left\{
	\begin {array}{ll}
		0 & \textrm{for $\sigma = \sigma'$}\\
		\alpha & \textrm{for $\sigma, \sigma' > 0$ and $\sigma \neq \sigma'$}\\
		\beta & \textrm{for $\sigma$ or $\sigma' = 0$}
	\end{array}
\right.
\end{eqnarray}
where $\alpha$ and $\beta$ are parameters for modelling the intercellular surface energy and the free surface energy, respectively. Previous studies have found that the magnitudes of these parameters have the effect on the cell shape and the flexibility of the cell boundaries: small magnitudes produce longer and rougher boundaries with more active cell dynamics, while large magnitudes result in straight boundaries and reduce the activeness of the cells \cite{szabo2010}.

The second term is an elastic constraint on the volume of each cell, since a physical cell has finite volume. $a_q$ and $A_q$ are the current and target volume of the ith cell. $λ$ sets the strength of this elastic constraint. Typical values used in simulating cellular monolayers are given in table ??. 

The bias w is a term proposed by Szabo et al. in their study of collective cell motion. 
\begin{eqnarray}
w (\bm{a} \rightarrow \bm{b}) = P \sum_{k \;=\; \sigma(a),\,\sigma(b)} \frac{\inc \bm{X}_k \cdot \bm{p}_k}{\abs{\bm{p}_k}}
\end{eqnarray}
where $\inc X_k$ is the change in the centre of cell labelled by spin $k$ in a elementary step, $p_k$ is a polarity vector that describes the preferred direction of the cell, and $P$ is a free parameter associated with the strength of the motility, which is proportional to the cell’s speed. The motivation for such 


\subsection{Motivation of the Study}
Recently, the paper by Bi and co-workers demonstrated that the jamming transitions in biological tissues can be simulated using the Self-Propelled Voronoi (SPV) model \cite{bimotility-driven2015}. In this model, each cell is represented as a e

In that study, researchers have found that the phase transition is characterised by three parameters: the speed of a single motile cell, the persistent time, and the target shape index. The

Given the success of CPM in modelling various cellular phenomena, it seems plausible that the model should also be capable of accounting the jamming transitions.  This is strengthened by the fact that the model has been used to study the collective behaviour of cell motility. 



\pagebreak
\section{Methodology}

\subsection{Code Structure}

%This section should contain the details of the method employed. 
%As in the previous sections standard techniques should not be written
%out in detail. For example if you use an oscilloscope to take a
%measurement, the theory of the CRO tube\footnote{Don't laugh, I have actually
%seen this.} is {\bf not relevant}. In computational projects this
%section should be used to explain the algorithms used and the layout of
%the computational code. A copy of the acutal code must be
%given in the appendices. Long detailed sections of theory, data tables
%and details of computational code used in data analysis only should not
%appear in this section, but should/may be included in the appendices.
%
%This section should emphasise the philosophy of the approach used
%and detail novel techniques. However
%please note: this section in {\bf not} a blow-by-blow account of what
%you did throughout the project, and in particular it should {\bf not} 
%contain large detailed sections about things you tried and found to be
%completely wrong. Remember you are writing a technical report, and
%not a diary. If however you find that a technique that was expected to
%work failed, that is a valid result and should be included.
%
%Here logical structure is particularly important, and you may find that
%to maintain good structure you may have to present the experiments
%in a different order from the one in which you carried them out.

\pagebreak
\section{Results and Discussion}

This section should detail the obtained results in a clear,
easy-to-follow manner. Remember long tables of numbers are just as boring to
read as they are to type-in. Use graphs to present your results where
-ever practicable. When quoting results or measurements
{\bf DO NOT FORGET ABOUT ERRORS}. Remember there are two basic types
of errors, these being random and systematic, which you must consider.
Remember also the difference between an error and a mistake, computer
program bugs are mistakes.

 
Again be selective in what you include. Half a dozen
tables that contain totally wrong data you collected while you forgot
to switch on the power supply are {\bf not relevant} and will frequently
mask the correct results. 
%
%                       Here is how to inserted a centered
%                       postscript file, this one is actually
%                       out of Maple, but it will work for other
%                       figures out of Xfig, Idraw and Xgraph
%
\begin{figure}[htb]     %Insert a figure as soon as possible
        \begin{center}
                \leavevmode             % Warn Latex a figure is comming
                \epsfxsize=90mm         % Horizontal size YOUR want
                                        % figure to be
                \epsffile{otf.eps}
\end{center}
\caption{This is an inserted Postscript file}
\end{figure}

This section must contain a discussion of the results. This should
include a discussion of the experimental and/or numerical errors, and a
comparison with the predictions of the background and theory underlying
the techniques used. This section should highlight particular strengths
and/or weaknesses of the methods used.

 
\section{Conclusion}
This section should summarise the results obtained, detail
conclusions reached, suggest future work, and changes that you would make if you repeated the
experiment. This section should in general be short, 100 to 150 words
being typical for most projects.
\par\noindent
If you have opted to have multiple {\bf Theory, Method, Results}
sections, draw all the results together in a {\bf single} conclusion.
\section{References}

Don't forget this section. Detail the relevant references which
should be cited at the correct place in the text of the report. There
are no fixed rules as to how many references are {\it needed}. Generally
the longer the project, and the more background reading you had to do,
the more references will be required. 

When you cite a reference you must give sufficient information. For
example, for a journal article give, {\it Author}, {\it Title of
article},
{\it Journal Name}, {\it Volumn}, {\it Page}, and {\it Year}, 
while for a book give, {\it Author}, {\it Title},
{\it (Editor if there is one)}, {\it Publisher}, and {\it Year}.        

\bibliographystyle{unsrt}   
\bibliography{bibliography}


\appendix
\section{Appendices}

%Material that is useful background to the report, but is not essential,
%or whose inclusion within the report  would detract from its
%structure and readablity, should be included in appendices. Typical
%material could be diagrams of electronic circuits built, specialist
%data tables used to analyse results, details of computer programs
%written for analysis and display of results, photographic plates,
%and, for computational projects, a copy of all written code.
%
%Again be selective. The appendix is {\bf not} an excuse for you to add every
%last detail and piece of data, but should be used to assist the reader
%of the report by supplying additional material. Not all reports require
%appendices and if the report is complete without this additional
%material leave it out.

\subsection{Simulation Code}

\end{document}
