%
%                       This is a LaTeX 2e version of the
%                       laboratory project template file.
\documentclass[a4paper,12pt]{article}
\usepackage{fullpage,epsf}
\usepackage{amsmath}
\usepackage{amsfonts}
\usepackage{amssymb}
\usepackage{amstext}
\usepackage{bm}
\usepackage{braket}
\usepackage{array}
\usepackage{graphicx}
\usepackage{tabularx}
\usepackage{url}
\usepackage{verbatim}
\usepackage{listings}
\usepackage{color}
\usepackage{courier}
\usepackage{epstopdf}
\usepackage{placeins}
\usepackage{subcaption}
\usepackage[font=footnotesize]{caption}
\usepackage[font=footnotesize]{subcaption}
\epstopdfsetup{update}
\usepackage[]{cite}


%%%%%%%%%%%%%%%%%%%%%%%%%%%%%%%%%%%%%%%%%%%%%%%%%%%%%
\renewcommand{\vec}[1]{\mathbf{#1}}
\newcommand{\abs}[1]{\left|#1\right|}
\newcommand{\inc}{\Delta}
\newcommand{\etal}{\emph{et al. }}
%%%%%%%%%%%%%%%%%%%%%%%%%%%%%%%%%%%%%%%%%%%%%%%%%%%%%

%
%                       This section generates a title page
%                       Edit only the sections indicated to put
%                       in the project title, your name, supervisor,
%                       project length in weeks and submission date
%
\begin{document}
\pagestyle{empty}                       % No numbers of title page                      
\epsfxsize=40mm                         % Size of crest
\begin{minipage}[b]{110mm}
        {\Huge\bf School of Physics\\ and Astronomy
        \vspace*{17mm}}
\end{minipage}
\hfill
\begin{minipage}[t]{40mm}               
        \makebox[40mm]{
        \includegraphics[width=4cm]{crest.jpg}}
\end{minipage}
\par\noindent                                           % Centre Title, and name
\vspace*{2cm}
\begin{center}
        \Large\bf \Large\bf Senior Honours Project\\
        \Large\bf Physics 4\\[10pt]                     % Change to MP/CP/Astro
        \LARGE\bf Investigation of Jamming in the Cellular Potts Model       % Change to suit
\end{center}
\vspace*{0.5cm}
\begin{center}
        \bf Michael Chiang\\                           % Repace with your name
        25 March 2015                                    % Submission Date
\end{center}
\vspace*{5mm}
%
%                       Insert your abstract HERE
%                       
\begin{abstract}
        The abstract is a short, concise explanation of the project
        covering the aims, outlines of techniques used and a short
        summary of the results. It should contain enough information to
        make the aims and success of the project clear, but contain no details.
        A typical abstract should be between 50 and 100 words.
\end{abstract}

\vspace*{1cm}

\subsubsection*{Declaration}

\begin{quotation}
        I declare that this project and report is my own work.
\end{quotation}

\vspace*{2cm}
Signature:\hspace*{8cm}Date: 25 March 2015

\vfill
{\bf Supervisor:} Prof. D. Marenduzzo                 % Change to suit
\hfill
10 Weeks                                         % Change to suit
\newpage
%
%                       End of Title Page
\pagestyle{plain}                               % Page numbers at bottom
\setcounter{page}{1}                            % Set page number to 1
\tableofcontents                                % Makes Table of Contents

\break
\section{Introduction}
Cell monolayers are widely studied biophysical systems as they are a simple model of a tissue. They provide an economic approach to study complex biological processes that take place in multi-cellular organisms. A particular research area where cell monolayers have been used as a model is in the study of dense biological tissues. These tissues, where cells are closely packed against each other with minimal spacing in between, are often found in important physiological and pathological processes such as embryonic development, wound healing, and cancer metastasis \cite{friedl2009}.  

Recent experiments have demonstrated that these dense tissues exhibit both fluid and glass-like behaviours \cite{angelini2010, schoetz2013}. This is reminiscent of the ``jamming'' transition observed in condensed matter, which describes the transition of matter with disordered structure from a fluid (unjammed) to rigid (jammed) state. This transition is widely studied in non-living systems such as colloidal dispersions, granular matter, foams, and emulsions. 

Computer simulations have been conducted to further understand the mechanics behind the fluid and rigid behaviours of tissues. To this end, multiple models have been developed to describe the dynamics of cells within tissues. A particular successful model is the vertex model, where cells are represented as a polygonal tiling of space and their dynamics are governed by a set of force equations on the vertices of the polygons \cite{nagai2001}. Recent computational studies indicate that the model exhibits a jamming transition in agreement with empirical observations, with cell shape and cell motility being the key parameters in determining the rigidity of the tissue \cite{bi2015density, bi2015motility}.

Another widely used model in simulating cell dynamics in tissues is the cellular Potts model (CPM). This model represents cells as domains on a lattice, and their movements are governed by an energy function that characterises the cellular interactions. There, however, has not been simulations conducted to verify whether the model has a fluid to rigid transition. With this in mind, the aim of this project is to implement the cellular Potts model and to investigate whether varying the model parameters would reproduce the jamming transition observed in dense tissues. 

The remaining sections of this paper are organised as follows: Section 2 provides a description of the CPM, the concept of jamming, and its connection to the dynamics in dense biological tissues. Section 3 outlines the implementation of the model and the experimental methods for investigating whether CPM can induce the jamming transition. Section 4 and 5 present and discuss the results obtained from the simulations. 


%The introduction section of the report should introduce the project in
%more detail than in the abstract. In particular it should present the
%motivation, the aims, outline of techniques used, and the scope of the project. 
%It should also contain references to similar work in the
%same field to put your work in the correct context.
%
%As a general rule, people reading the abstract and introduction alone
%should have a good idea of the material in the project, the techniques
%employed and the results obtained. A typical introduction should be
%about 1 page, (300-450 words). 


\section{Background and Theory}
\subsection{Cellular Potts Model}
\label{sec:CPM}
Developed by Glazier and Graner\cite{graner1992}, the cellular Potts model (CPM)\footnote{The CPM is also known as the GGH model, naming after Glazier and Graner who first proposed it and Hogeweg who extended it for various biological modelling applications.} is a stochastic model widely used in simulating the motion of closely packed cells in a monolayer. The model is an extension of the Potts model used in statistical physics and takes into account of the physical characteristics and dynamics of biological cells. The model has successfully explained collective behaviours such as cell sorting \cite{graner1992} and streaming\cite{szabo2010}.

In the CPM, a cell monolayer is represented on a two dimensional lattice. Each lattice site is assigned with a non-negative integer spin $\sigma$ known as the cell index, where $\sigma \in [0,N]$ and $N$ is the total number of cells. Two lattice sites which share a side or a corner are considered as neighbours. A biological cell is denoted by a set of connected sites with the same spin (see figure \ref{fig:CPM}). $\sigma = 0$ is reserved for enumerating sites that are not part of any cell. Throughout this project, periodic boundary conditions were applied such that the boundaries have no influence on the cells dynamics.

\begin{figure}[h]
\centering
\includegraphics[width=7.0cm]{figure/CPMDiagram.png}
\caption{A pictorial representation of the cellular Potts model. A cell is represented by a domain in the lattice, which is a set of interconnecting sites with the same spin value. Diagram taken from Glazier \etal, Single Cell Based Models in Biology and Medicine [cite].}
\label{fig:CPM}
\end{figure}

The dynamics of the cells are evolved using a modified version of the Metropolis algorithm: In the simulation, a series of elementary steps are taken with the goal of minimising an effective Hamiltonian $H$ that characterises the properties of cells and their interactions. Each step is an attempt to change the spin of a random lattice site $\bm{a}$ to that of a randomly chosen neighbour site $\bm{b}$, where $\sigma(\bm{a}) \neq \sigma(\bm{b})$. The probability of accepting a spin change, $p\left(\sigma(\bm{a}) \rightarrow \sigma(\bm{b})\right)$, is determined by the associated change in energy ($\inc H$): if $\inc H \leq 0$, the spin change is always accepted; if $\inc H > 0$, the spin change is accepted with a probability of $\exp(-\inc H / kT)$, where $k$ is the Boltzmann constant and $T$ is the effective temperature. These relations can be conveniently summarised by the following expression:
\begin{eqnarray}
\ln p\left(\sigma(\bm{a}) \rightarrow \sigma(\bm{b})\right) = \min\left[0,-\frac{\inc H}{kT}\right].
\end{eqnarray}
The exact form of the Hamiltonian will be discussed below. Since changing $k$ and $T$ is equivalent to multiplying the energy function by an irrelevant constant, it is conventional to set $k$ and $T$ to $1$ and express the energy in units of $kT$. As updating the entire lattice takes more steps for a larger system, the number of elementary steps taken is not a proper measure of time. It is conventional to define a unit time as $L^2$ number of spin change attempts, where $L$ is the linear dimension of the lattice. This is known as one Monte-Carlo step (MCS). One also defines the acceptance rate as the ratio of the number of spin copies accepted to the number of attempts made. This is important for an accurate comparison between the time scales used in simulations and those used in experiments\cite{sanz2010}.

As mentioned above, the system's dynamics are governed by an effective Hamiltonian that describes the physical behaviours of cells in tissues. This is given by:
\begin{eqnarray}
\label{eqn:hamiltonian}
H = \sum_{\langle{\bm{x}, \bm{x'}\rangle}} J\left(\sigma(\bm{x}), \sigma(\bm{x'})\right) + \lambda \sum_{i\,=\,1}^{N} \left(a_i - A_i\right)^2 - P \;\sum_{i\,=\,1}^{N} \bm{n}_i \cdot \bm{r}_i.
\end{eqnarray}
The first term accounts for the overall interfacial behaviours of cells. This includes cell-cell adhesion and the surface tension of each cell. The term is a sum over the interfacial energy for all pairs of neighbours in the lattice. $J$ describes the specific interfacial energy between two sites $\bm{x}$ and $\bm{x}'$. For simulating an ensemble of homogenous cells, $J$ is specified by the following values:
\begin{eqnarray}
\label{eqn:interfacial_energy}
J(\sigma,\sigma') = \left\{
	\begin {array}{ll}
		0 & \textrm{for $\sigma = \sigma'$}\\
		\alpha & \textrm{for $\sigma, \sigma' > 0$ and $\sigma \neq \sigma'$}\\
		\beta & \textrm{for $\sigma$ or $\sigma' = 0$},
	\end{array}
\right.
\end{eqnarray}
where $\alpha$ and $\beta$ are parameters for modelling the interfacial energy between two cells and the free boundary energy, respectively. Previous studies have found that the magnitudes of these parameters affect the cell shapes and the flexibility of the cell boundaries: small magnitudes produce long and rough boundaries with more active cell dynamics, while large magnitudes result in straight boundaries with less cellular motion \cite{szabo2010}.

The second term addresses the fact that cells are approximately the same size due to their incompressibility and resistance to height fluctuations. These aspects are modelled as an elastic constraint in the Hamiltonian. $a_i$ and $A_i$ are the current and target area of the $i$th cell. $\lambda$ sets the strength of this constraint and governs the degree of fluctuations of each cell's area.

The third term is proposed by Szabo \emph{et al.} in their study of collective streaming behaviours in cell monolayers \cite{szabo2010}. It captures the aspect that  cells can generate a motile force and move along a polarized direction. The term models this effect by favouring spin changes which result in the cells moving along their preferred direction. Within the term, $\bm{n}_i$ is a polarity vector that describes the preferred direction of the cell, $\bm{r}_i$ is the centre of mass of the cell, and $P$ is a free parameter associated with the strength of the motility (taken to be homogeneous). The specific mechanisms of how a cell generates a polarized motion and how the polarity changes over time remain poorly understood and are subject to intense investigations [cite]. A simple model, as suggested by Bi \emph{et al.} \cite{bi2015motility} and is implemented in this work, is to assume that the polarity vector undergoes rotational diffusion over time. If one defines the polarity angle of the $i$th cell ($\theta_i$) as the angle between the cell's polarity vector ($\bm{n}_i$) and a reference axis (taken to be the $x$-axis herein), the rotational diffusion process is governed by the following equations:
\begin{eqnarray}
\partial_t\theta_i(t) & = & \eta_i(t)\\
\langle{\eta_i(t)\eta_j(t')\rangle} & = & 2D_r\delta(t-t')\delta_{ij}
\end{eqnarray}
where $\eta_i$ is a noise function with zero mean and a variance of $2D_r$ with $D_r$ being the rotational diffusion coefficient. On a computer, the evolution of the polarity angle over time can be represented by:
\begin{eqnarray}
\label{eqn:computerRotateDiff}
\theta_i(t+\Delta t) = \theta_i(t) + \sqrt{2D_r\Delta t}\,\tilde\eta
\end{eqnarray}
where $\tilde\eta$ is a random number generated such that the last term reproduces the statistical mean and variance of the noise function. For simplicity, this can be achieved by generating a uniformly distributed random number between $\sqrt{3}\,[-1,1)$ , where the factor of $\sqrt{3}$ comes from normalisation\footnote{See Appendix \ref{app:rotatediff} for a detailed derivation of this relation.}.

\subsection{Jamming Transitions in Biological Tissues}
In condensed matter, jamming describes the transition from a fluid to a rigid state in which the structure of the material remains disordered. This occurs in various non-biological systems such as emulsions, foams, granular materials, and colloidal dispersions\cite{hecke2010}. For example, colloidal particles can jam and become rigid when the packing fraction is increased beyond a critical point, while foams can lose its ability to flow when the applied stress is lowered. The conditions for the transition are described by the ``jamming phase diagram,'' which was first proposed by Liu and Nagel (see figure \ref{fig:JammingPhaseDiagram}) \cite{liu1998}. The key parameters governing the transition are the temperature, the applied stress, and the density of the material. For instance, in the well-known glass transition, a fluid becomes solid-like but maintains a disordered structure when cooled. This corresponds to moving downwards along the temperature -- 1/density plane in the diagram. 

\begin{figure}[h]
\centering
\includegraphics[width=9.0cm]{figure/JammingPhaseDiagram.jpeg}
\caption{The jamming phase diagram proposed by Liu and Nagel. The glass transition corresponds to moving vertically in the temperature -- 1/density plane.}
\label{fig:JammingPhaseDiagram}
\end{figure}

Most epithelial and endothelial tissues, especially those in monolayers, have fluid-like properties. The cells within them do not have long range positional order and can slide past each other. Recent experiments have shown that these dense biological tissues also display glassy, rigid dynamics: Firstly, the tissues exhibit \emph{viscoelasticity} \cite{schoetz2013}. They behave like an elastic solid at short timescales but like a viscous fluid at longer timescales. The typical time for the system to transit from a solid to a fluid-like state is known as the relaxation time. Secondly, the trajectories of individual cells reflect the \emph{caging effect} \cite{schoetz2013}. This is the phenomenon in which individual elements of the system are trapped in a ``cage'' formed by their surrounding neighbours due to dense packing. Escaping the cage requires co-operative movements between the element and its neighbours, which only happens on rare occasions when the energy fluctuations are large. Thirdly, the cells also demonstrate \emph{dynamic heterogeneity} \cite{angelini2010}. This refers to the strong disparity in the movements and forces between individual constituents of the system over space and time. 

The exhibition of both fluid and glass-like behaviours in dense tissues have prompted investigations on the jamming transition in these systems. A major focus is on identifying the relevant parameters, analogous to those in the jamming phase diagram, that govern the transition. To this end, both experimental and computational approach have been used. Computer simulations have been particularly useful as they allow one to vary cell properties systematically, which would be difficult to achieve in experiments. Multiple models have been developed and extensively studied for characterising cell dynamics in tissues. Two widely-used models are the vertex model and the CPM.  

\subsection{Motivation of the Study}
Recently, Bi and co-workers demonstrated that the vertex model and its variant, the Self-Propelled Voronoi (SPV) model, exhibit the jamming transition observed in confluent cell monolayers, where there are no gaps between cells \cite{bi2015density, bi2015motility}. These models represent individual cells as polygons that fill a two dimensional space. The dynamics are controlled by a set of force equations on the vertices or, in the case of SPV, the centres of the polygons. Researchers have found that the system's rigidity is controlled by three properties: the cell shape, the cell motile strength, and the persistent time, which is the characteristic time a cell would travel before changing its direction. The cell shape is influenced by factors such as cell-cell adhesion and surface tension. A higher adhesion would lead to longer, more flexible cell shapes since it favours interfaces between cells. In the limit of zero motility, researchers have found that the cell shape alone can predict the transition point, with higher cell-cell adhesion, or longer interfaces, leads to fluid-like behaviours. When the cells are motile and have a longer persistent time, the rigidity is relaxed and less adhesion is required for cells to unjam.

A less extensive effort, however, has been done on investigating whether the jamming transition exists in the CPM. Although the work by Kabla \cite{kabla2012} and Szab\'o \cite{szabo2010} have hinted such a possibility, there has not been a specific study conducted to analyse jamming in the CPM.  This project, thus, aims to provide this analysis. As suggested by Bi \emph{et al.}, cellular interactions, which affect the cell shape, and cell motility are two key quantities in determining the transition point. These correspond well to the CPM parameters $\alpha$ and $P$, respectively. In light of this, there are two objectives in this project: Firstly, to determine whether varying the parameter $\alpha$, which models the interfacial effects of cells such as surface tension and cell-cell adhesion, has the ability to induce a jamming transition. Secondly, to understand the effect of altering the strength of cell motility, $P$, has on the rigidity of the system.


\section{Methodology}
The project is divided into two parts to achieve the objectives as discussed. The first part focuses on implementing the cellular Potts model on a computer. The second part is devoted to changing the model parameters systematically to investigate whether the model exhibits the jamming transition.

\subsection{Implementation of the Cellular Potts Model}
\subsubsection{Programming Language and Software Used}
The CPM was implemented in Java as it is a robust, platform-agnostic language. Java also provides an easy-to-use framework for developing a graphical user interface, which is needed in this project to visualise the cell motions. To facilitate the development process, the Eclipse integrated development environment was used, as it has a comprehensive code browsing and debugging toolkit. In addition, the version control system, Git, was used to maintain a history of the development and to facilitate the synchronisation of the code across multiple platforms. The ANT automated build system was also employed to allow rapid code compilation.

\subsubsection{Code Design and Structure}
The main structure and design of the code are described as follows: There are three main groups of classes in the program (see figure \ref{fig:ClassDiagram}). Firstly, there are the model classes that handle the main computation as specified by the CPM. These include the \texttt{CellPottsModel} and the \texttt{SpinModel} class. \texttt{CellPottsModel} contains most of the logic of the model, including the Metropolis algorithm for updating spins and the calculation of the various statistical quantities measured in the experiments. The \texttt{SpinModel} is the main interface by which other classes access data from the model. The second group is the visualisation classes which display the model to the user in a graphical user interface (see figure \ref{fig:CPMGUI} for the developed interface). These include the \texttt{CPMView}, \texttt{CPMViewPanel}, and the \texttt{CPMControlPanel} classes. The \texttt{CPMView} manages the operations related to the window frame. The \texttt{CPMViewPanel} visualises the model by drawing individual domains or cells on the screen with different colours as the simulation progresses. The \texttt{CPMControlPanel} handles the parameters supplied by the user for running the model.  A third group of classes is responsible for I/O processes. These include a set of \texttt{DataWriter} classes which record the statistical quantities measured at each time step of the simulation to text files. In addition, the \texttt{SpinReader} class is responsible for parsing any inital lattice configurations specified by the user to the program.  
\begin{figure}[h]
\centering
\includegraphics[width=0.85\textwidth]{figure/ClassDiagram.pdf}
\caption{A UML class diagram of the computer program developed which implements the CPM. Notice the program is divided into three groups of classes: the model classes, the visualisation classes, and the I/O classes. In addition, notice the implementation of the observer design pattern between the view and the model classes as a mean to reduce their coupling. The arrows in the diagram represent the association and navigability between classes: the class at the tail of the arrow has a pointer to the class at the arrow head.}
\label{fig:ClassDiagram}
\end{figure}
\begin{figure}[h]
\centering
\includegraphics[width=0.65\textwidth]{figure/CPMGUI.pdf}
\caption{Developed Graphical User Interface for the Cellular Potts Model}
\label{fig:CPMGUI}
\end{figure}
\FloatBarrier

The dependency between the model and visualisation classes is kept at a minimum to reduce the potential impact or changes required when one of them is modified. This is achieved by means of the observer design pattern. The model class extends the native Java class \texttt{Observable}, while the visualisation classes implement the \texttt{Observer} interface, which requires classes to implement the method \texttt{update()} that is called when the \texttt{Observable} has changed (see figure \ref{fig:ClassDiagram}). The \texttt{Observable} class manages the entire process of notifying the classes which implement the \texttt{Observer} interface when the model has changed. The actual model classes, therefore, do not need to have specific knowledge about the visualisation classes, so the coupling between the model and visualisation classes is minimised.

For more details about the responsibilities and functionalities of each class, please see the \texttt{Javadocs} included in the source code (see Appendix ?? for more information about accessing the source code). 



\subsubsection{Verification of the Implemented Model}
It is important to verify that the program implemented the model correctly before using it to perform experiments. This was achieved via two approaches: Firstly, a set of unit test cases were developed to test individual methods of the program to ensure the calculations are correct. These test cases are included in the source code package. Secondly, the program was tested against the results from other published papers which investigated the CPM. In particular, simulation data for a single cell suspended in a fluid are compared to those obtained by Szabo and coworkers \cite{szabo2010}. By using the same set of parameters (i.e. $\alpha = 2.0$, $\beta = 1.0$, $\lambda = 1.0$, $P = 0.0$, and cell target area $A = 50$) and averaging the results from 1000 trials, the root-mean-square displacement (see section 3.2.1) of the cell after 600 MCS was found to be $5 \pm 3$ pixels. This agrees, within the measurement error, to that reported by the researchers, which is approximately $3 \pm 1$ pixels. Therefore, there is a high degree of confidence that the written program has implemented the model accurately.


\subsection{Experimental Methods}
\subsubsection{Simulation Methods}
All simulations conducted were run on a $200 \times 200$ lattice with 1000 cells. The target area of the cells was set to be the average area occupied by each cell, which is 40 lattice sites. $\lambda$ and  $D_r$ were set to be 1.0 and 0.1, respectively. These parameters were chosen as they produce a physical representation of cells in tissues, and they were also used in other CPM simulations \cite{szabo2010, graner1992}. Notice the value for $\beta$ is irrelevant here since the focus is on confluent tissues, where there is no empty space between cells and hence no free boundaries. As mentioned above, periodic boundary conditions were assumed to eliminate the effects of external boundaries on the cell dynamics. 

For practical reasons, the same initial condition was used for all simulation. This should not result in correlation between measurements obtained in different trials, as the spin-update process is randomised and a different seed was used in each trial for the random number generator. The initial condition was prepared such that the system begins in a steady-state condition. This minimises the effect of transient behaviours on the measurements. To achieve this, cells were first initialise as squares with equal area on the lattice. The system was then allowed to run for 200000 MCS to reach steady-state. 

\subsubsection{Experiments Conducted}
Two experiments were performed to determine if jamming exists in the CPM. The first experiment investigates whether increasing the model parameter $\alpha$, which represents the interfacial energy between cells, would induce a jamming transition. Physically, this corresponds to an increase in the cells' surface tension or a reduction in the adhesiveness between cells. To testify this, 20 simulation trials were performed for each $\alpha$ ranging from 0.5 to 10.0 in increment of 0.5 with no cell motility. Each trial was run for 100000 MCS, which is a typical time range used in CPM simulations \cite{kabla2012}. Despite the typical $\alpha$ values explored by researchers are between 0.5 to 4.0, higher  values were considered in this project to obtain a complete picture of the parameter?s effect on the system. Furthermore,  $\alpha$'s lower than 0.5 were not considered as cells tend to dissociate in this range. 

The second experiment investigates whether increasing the cell motility strength,, would relax the rigidity of the system as observed in the vertex model. The variation of  was explored in systems with different ?s ranging between 1.0 to 4.0 in intervals of 0.2. For each $\alpha$, $P$ was systematically increased from 0.0 to 5.0 in increments of 0.2. 10 simulation trials were conducted for each set of  $\{\alpha,P\}$, and each trial was run for 10000 MCS. The number of trials and simulation time were reduced for this experiment due to the substantial increase in computation required when motility is induced, as one needs to calculate the centre of mass of the affected cells in every elementary step (see equation ??). 

\subsubsection{Statistics Measured in Simulations}
\label{sec:StatsMeasure}
An important step towards determining the existence of jamming in the CPM is to identify whether the system exhibits rigid, glassy dynamics. Two statistical quantities widely used in studying glass-like behaviours are the mean square displacement and the non-Gaussian parameter. These were measured in all simulation trials. 

\paragraph{Mean Square Displacment}
The mean square displacement (MSD) of the system at a particular time $t$ is given by:
\begin{equation}
\langle{\bm{r}^2(t)\rangle} = \langle{\left(\bm{r}_i(t) - \bm{r}_i(0)\right)^2\rangle},
\end{equation}
where $\bm{r}_i (t)$ is the centre of mass of the $i$th cell at the time $t$ and $\langle...\rangle$ denotes an ensemble average over all cell trajectories of the system. The MSD measures the deviation of the cells from their initial positions. In the fluid regime, the cells are diffusive and their MSD is expected to increase linearly with time:\begin{eqnarray}
\langle{\bm{r}^2(t)\rangle} = 2dDt, 
\end{eqnarray}
where $D$ is the diffusion coefficient or the diffusivity and $d$ is the number of dimensions of the system, which is 2 in this case. In the glass-like regime, cells movements are restricted due to the cage effect (section 2.2). This causes sub-diffusive behaviours at intermediate time scales when cells remain trapped in their cages, and the MSD typically shows a power-law relationship with time where the exponent is smaller than one [Hofling]. This effect is indicated by a plateauing region when plotting the MSD curve against time on a logarithmic scale (see figure 4).
\begin{figure}[h]
\centering
\includegraphics[width=0.5\textwidth]{figure/MSDExpectedCurve.pdf}
\caption{The typical MSD versus time graph on a log-log plot. Notice the plateau region due to the sub-diffusive movement of the cells because of the caging effect. Diagram take from: \cite{hofling2013}.}
\label{fig:MSDExpectedCurve}
\end{figure}

\paragraph{Non-Gaussian Parameter}
The non-Gaussian parameter $\alpha_2$ measures the deviation from a Gaussian form in the van Hove self-correlation function, which is related to the probability distribution of a particle's displacement from its original position after a specific time [Rahman1964] [Hofling]. Mathematically, the parameter is the kurtosis of the correlation function and, in two dimensions, is given by:
\begin{equation}
\alpha_2 = \frac{1}{2}\,\frac{\langle{\bm{r}^4(t)\rangle}}{\langle{\bm{r}^2(t)\rangle}^2} - 1,
\end{equation}
where $\langle{\bm{r}^4(t)\rangle} \equiv \langle{\left(\bm{r}_i(t) - \bm{r}_i(t_0)\right)^4\rangle}$. The parameter is zero when the probability distribution is Gaussian, which is the case when the system is diffusive in the fluid regime. The motivation for measuring this parameter is to detect the cage breaking events that occur in glass-like regime at intermediate time scales [Schotz]. These events would cause the probability distribution to become non-Gaussian as cells would have more directed motion when changing neighbours.

\subsubsection{Classification of Fluid and Solid-Like Behaviour}
To quantitatively distinguish the system?s behaviour between the fluid and solid regime, the diffusivity and the exponent of the MSD were measured. These quantities were also used to analyse whether the transition from fluid to solid, if exists, is a smooth crossover or a critical phenomenon. 

\paragraph{Diffusivity}
The diffusivity $D$ measures the ability of the system to diffuse. In two dimensions, it is defined by:
 \begin{eqnarray}
D = \frac{1}{4}\, \lim_{t \rightarrow \infty} \frac{\langle{\bm{r}^2(t)\rangle}}{t}.
\end{eqnarray}
If the system is deep in the jammed state, the cells would remain frozen for a long period of time due to confinement by their neighbours. This results in a low diffusivity. In the limit where the system is completely rigid and sub-diffusive behaviour (i.e. MSD scales sub-linearly with time) persists for infinitely long, the diffusivity is zero. Therefore, one can argue that the fluid-to-solid transition occurs when the system?s diffusivity falls below a certain threshold limit. This quantity was used by Bi et al. as part of their analysis in determining the transition point in the vertex model [Bi2015].

There are two common approaches in calculating the diffusivity of the system. One approach is to perform a linear least squares fit at a timescale beyond the sub-diffusive interval where the mean square displacement is linear. Another approach uses the definition of diffusivity. As the simulation time for each trial is at least $10^4$ MCS, one can approximate the diffusivity by using the mean square displacement value from the final time step $t_f$:
\begin{eqnarray}
D \approx \frac{1}{4} \frac{\langle{\bm{r}^2(t_f)\rangle}}{t_f}.
\end{eqnarray}
Both methods were used in the project to find the diffusivity of the system. The obtained results are compared in the section 4.1.

\paragraph{Diffusion Exponent}
As mentioned above, systems with glassy dynamics exhibit sub-diffusive behaviours at intermediate time scales during which the MSD has a power-law relationship with time. Therefore, a useful parameter to classify whether the system has a fluid or solid behaviour is the exponent $\gamma$ in the power law, which is given by:
\begin{equation}
\langle\bm{r}^2(t)\rangle \sim t^{\gamma}
\end{equation}
At a specific time, one can consider the system to be fluid-like if the cells are diffusive and $\gamma = 1$. On the other hand, one can classify the system to be in the solid, glassy regime if the cells are sub-diffusive and $\gamma < 1$.

\section{Results}
\subsection{Effects of Varying Cellular Interaction ($\alpha$)}
\begin{figure}[h]
\centering
\includegraphics[width=0.8\textwidth]{figure/R2_Alpha.pdf}
\caption{Mean square displacement of the cells versus time on a log-log plot for $\alpha$ = 0.5 (top green curve) to 10 (bottom yellow curve) in increments of 0.5. The typical intercellular separation (squared) is also plotted as a dashed-line to indicate whether the cells have exchanged neighbours. Notice the difference between the final MSD value for $\alpha = 1.0$ (light blue curve) and that for  $\alpha = 1.5$ (orange curve) is more than decade, which is larger than the difference between the final MSD values for higher $\alpha$.
}
\label{fig:r2alpha}
\end{figure}

Figure \ref{fig:r2alpha} shows the MSD of the cells versus time on a logarithmic scale for $\alpha$ between 0.5 and 10.0 in increments of 0.5. As $\alpha$ increases, the displacement of the cells become suppressed. The MSD curves flatten in the time interval between 10 and 1000 MCS for $\alpha > 5$. Furthermore, for $\alpha$ between 1.5 and 4, the curves start to bend when $t \sim 100$ MCS. The plateauing behaviour indicates the cells are sub-diffusive. This can be explained by the cage effect where cells are confined by their neighbours and have difficulty to diffuse. 

The slopes of the MSD curves also suggest that the system exhibits two types of dynamics. For $\alpha < 1.5$, the cells are diffusive as one can qualitatively see the MSD increases linearly with time (the actual exponents are measured and will be discussed further below). For $\alpha > 1.5$, the system remains sub-diffusive even by the end of the simulation as indicated by the plateauing behaviour. It should also be noted that the difference between the final points of the MSD for $\alpha = 1.0$ and $\alpha = 1.5$ is more than a decade, which is much larger than the difference between those for higher $\alpha$ values. These observations suggest that there is a change in the dynamics of the system at around $\alpha = 1.5$, with the system behaving like a fluid below this point. 

It is also possible to estimate whether the system has a fluid or solid-like dynamics by comparing the average displacement of the cells to the typical separation between two cells (i.e. the intercellular distance). When the system is fluid-like, one would expect the cells to have travelled, on average, more than the intercellular distance over the simulation period, as they would have experienced multiple neighbour exchange events. Since the target area of each cell is 40 $\textrm{pixel}^2$, cells are likely to have changed neighbours when the MSD surpasses this threshold, which is indicated by a dashed line in the figure. 
\begin{figure}[h]
\centering
\begin{subfigure}[h]{0.49\textwidth}
\includegraphics[width=\textwidth]{figure/CM_Alpha_1.pdf}
\caption*{$\alpha = 1.0$}
\end{subfigure}
\begin{subfigure}[h]{0.49\textwidth}
\includegraphics[width=\textwidth]{figure/CM_Alpha_2.pdf}
\caption*{$\alpha = 2.0$}
\end{subfigure}
\begin{subfigure}[h]{0.49\textwidth}
\includegraphics[width=\textwidth]{figure/CM_Alpha_5.pdf}
\caption*{$\alpha = 5.0$}
\end{subfigure}
\begin{subfigure}[h]{0.49\textwidth}
\includegraphics[width=\textwidth]{figure/CM_Alpha_10.pdf}
\caption*{$\alpha = 10.0$}
\end{subfigure}
\caption{The recorded trajectories of a sample of 100 cells in the simulation. }
\label{fig:CM}
\end{figure}
 It can be seen that only the MSDs for $\alpha$ between 0.5 and 1.5 go beyond this limit, suggesting that the system behaves like a fluid in this $\alpha$ range. This is consistent with the qualitative analysis on the slopes of the MSDs. 

Moreover, the observations on the MSDs are in agreement with the measured trajectories of the cells, which are shown in figure \ref{fig:CM}. When $\alpha = 1.0$, the trajectories overlap with each other to the extent that it is difficult to identify any structural pattern. This indicates that the system is fluid-like. When $\alpha = 2.0$, most trajectories become localised and a hexagonal structure starts to emerge, suggesting the system has solid-like behaviours. This agrees with the conjecture that there is a change in the system?s dynamics at around $\alpha = 1.5$. For $\alpha = 5.0$ and $\alpha = 10.0$, the trajectories are further suppressed, supporting the view that the system becomes more solid-like as $\alpha$ increases.  

\begin{figure}[h]
\centering
\includegraphics[width=0.8\textwidth]{figure/A2_Alpha_All.pdf}
\caption{The non-Gaussian parameter $\alpha_2$ of the cells versus time on a semi-log plot for $\alpha$ ranging from 0.5 (the green line at the top) to 10.0 (the yellow line at the bottom) in increments of 0.5. As $\alpha$ increases, the dynamics of the cells are slowed down.}
\label{fig:a2alpha}
\end{figure}

Figure \ref{fig:a2alpha} shows the measured non-Gaussian parameters for the studied $\alpha$ values. The behaviours of the parameters over time are consistent with the MSD data. The maximum of the parameter for each ? occurs at the time interval when the corresponding MSD plateaus. This is expected since sub-diffusive motion would necessarily result in a non-Gaussian probability distribution for the cell?s displacement. Moreover, the parameter has a higher maximum as $\alpha$ increases. This is consistent with the observation that the system becomes more sub-diffusive as $\alpha$ rises. 

As discussed in section 3.3, the motivation of measuring the non-Gaussian parameter is to identify the cage escape effect, which is a signature of glassy dynamics. However, as illustrated in figure \ref{fig:r2alpha}, the cells have moved an average distance that is less than the intercellular distance when $\alpha \ge 2.0$. This suggests that the non-Gaussian nature of the probability distribution cannot be attributed to the anticipated effect. Further investigation on individual cell trajectories would be needed to understand the origin of this non-Gaussian aspect. 



With both the MSD and the non-Gaussian parameter indicating the presence of glass-like, solid regime in the model, the analysis is focused on understanding whether the change from the fluid to a solid regime is a smooth crossover or a critical phenomenon. This was done by analysing the diffusivity and the exponent of the MSD obtained for the ? values and inv

Figure 8 shows the diffusivity of the system for the investigated ? values using both methods as discussed in section 3.4. The results obtained are in agreement for ? below 1.0 when the cells the diffusivity is non-negligible (D > 0.02). Both methods show that the diffusivity decreases as ? increases, especially in the region between 0.5 to 2.0. This is consistent with the picture developed from analysing the mean square displacement curves, that there is a change from fluid to solid behaviour in this region. However, the lack

Beyond ? = 1.5, the obtained diffusivity differs by around 50\% between the two approaches (see inset of figure 8). This is not surprising since the cell movements are sub-diffusive. The endpoint approach would definitely overestimate the diffusivity as it effectively measures the slope of a straight line from the origin to that point, which is clearly not the case as seen from the MSD figure. This suggest that the standard approach in determi

A more accurate approach in quantifying the state behaviour of the system is to consider the exponent of the MSD (?), which directly describes the degree of sub-diffusiveness of the system. Figure 9 shows the exponents for the studied ? values, which were obtained by performing least squares fit on the MSD curves in the time interval from t = 10000 to 50000 MCS. It provides a much more conclusive indication that there is a critical behaviour between 1.0 to 2.0. ? is near 1.0 for both ? = 0.5 and 1.0. However, when ? increases to 1.5, ? drops rapidly to 0.5 and remains within the range between 0.3 to 0.7 for higher ? values. This supports the view that the system is in a fluid regime when ? is below 1.5 and in a rigid, glassy regime when ? is above this value. The sudden, non-monotonic decrease in the exponent also indicate 

The fact that ? settles at around 0.5 is an interesting behaviour, as it coincides with the exponent predicted for systems undergoing single-file diffusion (SFD). This is the phenomenon when the system is rigid but has a porous which allow the movements of a single stream of particles or elements of the system. 
\begin{figure}[h]
\centering
\includegraphics[width=0.9\textwidth]{figure/Diffusivity_Combined_Alpha.pdf}
\caption{The measured diffusion exponent $\gamma$ for $\alpha$ ranging from 0.5 to 10.0 in an increment of 0.5. Error bars are plotted for the data points but are too small to be seen. Notice the abrupt decrease in the exponent between $\alpha = 1$ and $2$.  indicates that }
\end{figure}
\FloatBarrier

\begin{figure}[h]
\centering
\includegraphics[width=0.9\textwidth]{figure/MSDExpAvg_Alpha.pdf}
\caption{The measured diffusion exponent $\gamma$ for $\alpha$ ranging from 0.5 to 10.0 in an increment of 0.5. Error bars are plotted for the data points but are too small to be seen. Notice the abrupt decrease in the exponent between $\alpha = 1$ and $2$.  indicates that }
\end{figure}
\FloatBarrier

\subsection{Effects of Inducing Cell Motility $P$}

\begin{figure}[h]
\centering
\begin{subfigure}[h]{0.49\textwidth}
\includegraphics[width=\textwidth]{figure/R2_P_1_with_text.pdf}
\caption*{$\alpha = 1.0$}
\end{subfigure}
\begin{subfigure}[h]{0.49\textwidth}
\includegraphics[width=\textwidth]{figure/R2_P_2_with_text.pdf}
\caption*{$\alpha = 2.0$}
\end{subfigure}
\end{figure}

\begin{figure}[h]
\centering
\begin{subfigure}[h]{0.49\textwidth}
\includegraphics[width=\textwidth]{figure/CM_P_0.pdf}
\caption*{$P = 0.0$}
\end{subfigure}
\begin{subfigure}[h]{0.49\textwidth}
\includegraphics[width=\textwidth]{figure/CM_P_5.pdf}
\caption*{$P = 5.0$}
\end{subfigure}
\end{figure}
\begin{figure}[h]
\centering
\includegraphics[width=0.7\textwidth]{figure/Alpha_P_PhaseDiagram.pdf}
\caption{}
\end{figure}
\FloatBarrier


\section{Discussion}
%This section must contain a discussion of the results. This should
%include a discussion of the experimental and/or numerical errors, and a
%comparison with the predictions of the background and theory underlying
%the techniques used. This section should highlight particular strengths
%and/or weaknesses of the methods used.



\section{Conclusion}
%This section should summarise the results obtained, detail
%conclusions reached, suggest future work, and changes that you would make if you repeated the
%experiment. This section should in general be short, 100 to 150 words
%being typical for most projects.
%\par\noindent
%If you have opted to have multiple {\bf Theory, Method, Results}
%sections, draw all the results together in a {\bf single} conclusion.
%\section{References}
%
%Don't forget this section. Detail the relevant references which
%should be cited at the correct place in the text of the report. There
%are no fixed rules as to how many references are {\it needed}. Generally
%the longer the project, and the more background reading you had to do,
%the more references will be required. 
%
%When you cite a reference you must give sufficient information. For
%example, for a journal article give, {\it Author}, {\it Title of
%article},
%{\it Journal Name}, {\it Volumn}, {\it Page}, and {\it Year}, 
%while for a book give, {\it Author}, {\it Title},
%{\it (Editor if there is one)}, {\it Publisher}, and {\it Year}.        

%\bibliographystyle{unsrt}   
%\bibliography{bibliography}
\begin{thebibliography}{99}
	\bibitem{bi2015motility}
		D. Bi \etal
		Motility-driven glass and jamming transitions in biological tissues.
		\emph{arXiv}
		(2015)
	
	\bibitem{sanz2010}
		E. Sanz, D. Marenduzzo.
		Dynamic Monte Carlo versus Brownian dynamics: A comparison for self-diffusion and crystallization in colloidal fluids.
		\emph{J. Chem. Phys.}
		\textbf{132}, 194102 (2010)
	
	\bibitem{szabo2010}
		A. Szabo \etal
		Collective cell motion in endothelial monolayers.
		\emph{Phys. Biol.}
		\textbf{7} 046007 (2010)
		
	\bibitem{schoetz2013}
		E.-M. Schoetz \etal
		Glassy dynamics in three-dimensional embryonic tissues.
		\emph{J. R. Soc. Interface}
		\textbf{10} 20130726 (2013)
	
	\bibitem{angelini2010}
		T. E. Angelini \etal
		Glass-like dynamics of collective cell migration.
		\emph{Proc. Natl. Acad. Sci. U.S.A}
		\textbf{108} 4714 (2011)
	
	\bibitem{hecke2010}
		M. Hecke
		Jamming of soft particles: geometry, mechanics, scaling and isostaticity.
		\emph{J. Phys.: Condens. Matter}
		\textbf{22} 033101 (2010)
	
	\bibitem{liu2010}
		A. J. Liu, S. R. Nagel
		The jamming transition and the marginally jammed solid.
		\emph{Annu. Rev. Condens. Matter Phys.}
		\textbf{1} 347 (2010)
		
	\bibitem{graner1992}
		F. Graner, J. A. Glazier
		Simulation of biological cell sorting using a two-dimensional extended Potts model.
		\emph{Phys. Rev. Lett.}
		\textbf{69} 2013 (1992)
	
	\bibitem{bi2015density}
		D. Bi \etal
		A density-independent rigidity transition in biological tissues.
		\emph{Nat. Phys.}
		\textbf{11} 1074 (2015)
		
	\bibitem{kabla2012}
		A. J. Kabla
		Collective cell migration: leadership, invasion and segregation.
		\emph{J. R. Soc. Interface}
		\textbf{9} 3268 (2012)
	
	\bibitem{nagai2001}
		T. Nagai, H. Honda
		A dynamic cell model for the formation of epithelial tissues.
		\emph{Philos. Mag. Part B}
		\textbf{81} 699 (2001)
	
	\bibitem{friedl2009}
		P. Friedl, D. Gilmour
		Collective cell migration in morphogenesis, regeneration and cancer.
		\emph{Nat. Rev. Mol. Cell Biol.}
		\textbf{10} 445 (2009)
	
	\bibitem{liu1998}
		A. J. Liu, S. R. Nagel
		Nonlinear dynamics: Jamming is not just cool any more.
		\emph{Nature}
		\textbf{396} 21 (1998)
		
	\bibitem{hofling2013}
		F. H\"ofling, T. Franosch
		Anomalous transport in the crowded world of biological cells.
		\emph{Rep. Prog. Phys.}
		\textbf{76} 046602 (2013)
	
	

\end{thebibliography}

\appendix
\section{Appendices}
\subsection{Implementing Rotational Diffusion of the Cells on Computer}
\label{app:rotatediff}
In section \ref{sec:CPM}, the implementation for the rotational diffusion of the cell's polarity vector was discussed. The rotational diffusion process can be described by the following Langevin equations:
\begin{eqnarray}
\partial_t\theta_i(t) & = & \eta_i(t)\label{eqn:langevin}\\
\langle{\eta_i(t)\eta_j(t')\rangle} & = & 2D_r\delta(t-t')\delta_{ij}.
\end{eqnarray}
One can integrate equation \ref{eqn:langevin} to give:
\begin{eqnarray}
\theta_i(t+\inc t) - \theta_i(t) = \int_{t}^{t+\inc t} \eta_i (t')\,dt'.
\end{eqnarray}
The mean of the change in $\theta_i$ is:
\begin{eqnarray}
\langle\theta_i(t+\inc t) - \theta_i(t)\rangle = \int_{t}^{t+\inc t} \langle\eta_i (t')\rangle\,dt' = 0.\\
\end{eqnarray}
Similarly, the variance is:
\begin{eqnarray}
\langle\left(\theta_i(t+\inc t) - \theta_i(t)\right)^2\rangle & = & \int_{t}^{t+\inc t}\int_{t}^{t+\inc t} \langle\eta_i (t')\eta_i(t'')\rangle\,dt'\,dt''\\
& = & 2D_r \int_{t}^{t+\inc t}\int_{t}^{t+\inc t} \delta(t'-t'')\,dt'\,dt''\\
& = & 2D_r\inc t\label{eqn:anglevar}.
\end{eqnarray}
To implement this rotational diffusion process on a computer, one can use a random generator to produce a distribution of values that gives the same statistical mean and variance when evaluating the change in the polarity angle for each cell over many times. The simplest way to achieve this is to generate a uniform distribution of random values between $[-a,a)$, as this will satisfy the requirement of the mean automatically. The value $a$ is yet to be determined to give the required variance. The polarity angles of the cells can be updated by the following expression on the computer:
\begin{eqnarray}
\theta_i (t + \inc t) = \theta_i (t) + \sqrt{2 D_r \inc t}\, \tilde\eta,
\end{eqnarray}
where $\tilde\eta$ is the random number generated and $\inc t$ can be taken as the number of Monto-Carlo steps between each update (which is taken to be one in the written program). With this expression, one can determine $a$ to give the correct variance for the distribution of the change in polarity angles as specified by equation \ref{eqn:anglevar}. Since the random values are uniformly distributed, normalisation requires the probability density for each random value ($p(\tilde\eta)$) to be:
\begin{eqnarray}
p(\tilde\eta) = \frac{1}{2a}.
\end{eqnarray}
One can evaluate the actual variance produced using this distribution of random values:
\begin{eqnarray}
\langle\left(\theta_i(t+\inc t) - \theta_i(t)\right)^2\rangle & = & 2D_r\inc t \int_{-a}^{a} \tilde\eta^2 p(\tilde\eta)\, d\tilde\eta\\
& = & 2D_r\inc t\,\frac{1}{2a}\left[\frac{\tilde\eta^3}{3}\right]_{-a}^{a}\\
& = & 2D_r\inc t\,\frac{a^2}{3}.\\
\end{eqnarray}
Equating this with the expected variance in the change in polarity angle (see equation \ref{eqn:anglevar}), one must have:
\begin{eqnarray}
2D_r \inc t & = & 2D_r\inc t\,\frac{a^2}{3}\\
a & = & \sqrt{3},
\end{eqnarray}
so the random values must be between $\tilde\eta \in \sqrt{3}[-1,1)$.

\subsection{Additional Figures}


\subsection{Accessing the Source Code}
Due to the substantial size of the program, a copy of the source code is not included in the appendix of the report. The source code is available on GitHub with the following link: 

To download a copy of the source code, 

Please contact the author if a 
\end{document}
