%
%                       This is a LaTeX 2e version of the
%                       laboratory project template file.
\documentclass[a4paper,12pt]{article}
\usepackage{fullpage,epsf}
\usepackage{amsmath}
\usepackage{amsfonts}
\usepackage{amssymb}
\usepackage{amstext}
\usepackage{bm}
\usepackage{braket}
\usepackage{array}
\usepackage{graphicx}
\usepackage{tabularx}
\usepackage{url}
\usepackage{verbatim}
\usepackage{listings}
\usepackage{color}
\usepackage{courier}
\usepackage{epstopdf}
\usepackage{placeins}
\epstopdfsetup{update}
\usepackage[]{cite}


%%%%%%%%%%%%%%%%%%%%%%%%%%%%%%%%%%%%%%%%%%%%%%%%%%%%%
\renewcommand{\vec}[1]{\mathbf{#1}}
\newcommand{\abs}[1]{\left|#1\right|}
\newcommand{\inc}{\Delta}

%%%%%%%%%%%%%%%%%%%%%%%%%%%%%%%%%%%%%%%%%%%%%%%%%%%%%

%
%                       This section generates a title page
%                       Edit only the sections indicated to put
%                       in the project title, your name, supervisor,
%                       project length in weeks and submission date
%
\begin{document}
\pagestyle{empty}                       % No numbers of title page                      
\epsfxsize=40mm                         % Size of crest
\begin{minipage}[b]{110mm}
        {\Huge\bf School of Physics\\ and Astronomy
        \vspace*{17mm}}
\end{minipage}
\hfill
\begin{minipage}[t]{40mm}               
        \makebox[40mm]{
        \includegraphics[width=4cm]{crest.jpg}}
\end{minipage}
\par\noindent                                           % Centre Title, and name
\vspace*{2cm}
\begin{center}
        \Large\bf \Large\bf Senior Honours Project\\
        \Large\bf Physics 4\\[10pt]                     % Change to MP/CP/Astro
        \LARGE\bf Simulating the Jamming Transitions of a Cellular Monolayer with the Cellular Potts Model        % Change to suit
\end{center}
\vspace*{0.5cm}
\begin{center}
        \bf Michael Chiang\\                           % Repace with your name
        25 March 2015                                    % Submission Date
\end{center}
\vspace*{5mm}
%
%                       Insert your abstract HERE
%                       
\begin{abstract}
        The abstract is a short, concise explanation of the project
        covering the aims, outlines of techniques used and a short
        summary of the results. It should contain enough information to
        make the aims and success of the project clear, but contain no details.
        A typical abstract should be between 50 and 100 words.
\end{abstract}

\vspace*{1cm}

\subsubsection*{Declaration}

\begin{quotation}
        I declare that this project and report is my own work.
\end{quotation}

\vspace*{2cm}
Signature:\hspace*{8cm}Date: 25 March 2015

\vfill
{\bf Supervisor:} Prof. D. Marenduzzo                 % Change to suit
\hfill
10 Weeks                                         % Change to suit
\newpage
%
%                       End of Title Page
\pagestyle{plain}                               % Page numbers at bottom
\setcounter{page}{1}                            % Set page number to 1
\tableofcontents                                % Makes Table of Contents

\break
\section{Introduction}

%The introduction section of the report should introduce the project in
%more detail than in the abstract. In particular it should present the
%motivation, the aims, outline of techniques used, and the scope of the project. 
%It should also contain references to similar work in the
%same field to put your work in the correct context.
%
%As a general rule, people reading the abstract and introduction alone
%should have a good idea of the material in the project, the techniques
%employed and the results obtained. A typical introduction should be
%about 1 page, (300-450 words). 

\pagebreak
\section{Background and Theory}

%This section should cover the theory of the material in the project
%in sufficient detail to make the following work understandable to the
%average physicist. It should not contain large sections of standard
%bookwork, but should contain references to this material. The exact
%contents of this section will depend on the project being undertaken.
%
%This section should contain only the
%relevant theory. In particular a life history of the inventor of the
%technique to be used is
%totally irrelevant\footnote{I have seen a report that contained three pages
%on the life of Gabor, and it was not very interesting.}. Here use common sense
%and the general rule, ``If in doubt: leave it out'', however
%include information that you judge would be useful to one of your
%peers if they wehe to repeat the project. If you are
%undertaking a 12 week project and it includes a literature search, put the
%result of the search here. As a rough guide this section should be
%about 3-4 pages for a 6 week project, longer for longer projects.
%
%Note that if the project consists of a series of short experiments
%each of which requires a different theory and method, it may be appropriate
%to have one {\bf Theory, Method, Results} section for each
%experiment.

\subsection{Jamming Transitions in Biology}

The jamming\cite{bimotility-driven2015}

\subsection{Cellular Potts Model}


The cellular Potts model (CPM) is a statistical model that has been widely used to simulate the motion of closely packed cells. The model is an extension of th

CPM has been used for studying biological processes such as cell-sorting,

In a two-dimensional CPM, each lattice site is assigned with a non-negative integer spin $\sigma$ where $\sigma = 1, 2, ..., N$. A biological cell is associated with the collection of 
\begin{eqnarray}
H = \sum_{\langle{\bm{x}, \bm{x'}\rangle}} J\left(\sigma(\bm{x}), \sigma(\bm{x'})\right) + \lambda \sum_{q\,=\,1}^{N} \left(a_q - A_q\right)^2,
\end{eqnarray}
where the first term of Hamiltonian is associated with the surface energy that arises from the intercellular boundary

\subsection{Motivation of the Study}

\begin{eqnarray}
w (\bm{a} \rightarrow \bm{b}) = P \sum_{k \;=\; \sigma(a),\,\sigma(b)} \frac{\inc \bm{X}_k \cdot \bm{p}_k}{\abs{\bm{p}_k}}
\end{eqnarray}

\begin{eqnarray}
\ln p (\bm{a} \rightarrow \bm{b}) = \min \left[0, -\inc u (\bm{a} \rightarrow \bm{b}) + w (\bm{a} \rightarrow \bm{b}) \right]
\end{eqnarray}

\pagebreak
\section{Methodology}

\subsection{Code Structure}

%This section should contain the details of the method employed. 
%As in the previous sections standard techniques should not be written
%out in detail. For example if you use an oscilloscope to take a
%measurement, the theory of the CRO tube\footnote{Don't laugh, I have actually
%seen this.} is {\bf not relevant}. In computational projects this
%section should be used to explain the algorithms used and the layout of
%the computational code. A copy of the acutal code must be
%given in the appendices. Long detailed sections of theory, data tables
%and details of computational code used in data analysis only should not
%appear in this section, but should/may be included in the appendices.
%
%This section should emphasise the philosophy of the approach used
%and detail novel techniques. However
%please note: this section in {\bf not} a blow-by-blow account of what
%you did throughout the project, and in particular it should {\bf not} 
%contain large detailed sections about things you tried and found to be
%completely wrong. Remember you are writing a technical report, and
%not a diary. If however you find that a technique that was expected to
%work failed, that is a valid result and should be included.
%
%Here logical structure is particularly important, and you may find that
%to maintain good structure you may have to present the experiments
%in a different order from the one in which you carried them out.

\pagebreak
\section{Results and Discussion}

This section should detail the obtained results in a clear,
easy-to-follow manner. Remember long tables of numbers are just as boring to
read as they are to type-in. Use graphs to present your results where
-ever practicable. When quoting results or measurements
{\bf DO NOT FORGET ABOUT ERRORS}. Remember there are two basic types
of errors, these being random and systematic, which you must consider.
Remember also the difference between an error and a mistake, computer
program bugs are mistakes.

 
Again be selective in what you include. Half a dozen
tables that contain totally wrong data you collected while you forgot
to switch on the power supply are {\bf not relevant} and will frequently
mask the correct results. 
%
%                       Here is how to inserted a centered
%                       postscript file, this one is actually
%                       out of Maple, but it will work for other
%                       figures out of Xfig, Idraw and Xgraph
%
\begin{figure}[htb]     %Insert a figure as soon as possible
        \begin{center}
                \leavevmode             % Warn Latex a figure is comming
                \epsfxsize=90mm         % Horizontal size YOUR want
                                        % figure to be
                \epsffile{otf.eps}
\end{center}
\caption{This is an inserted Postscript file}
\end{figure}

This section must contain a discussion of the results. This should
include a discussion of the experimental and/or numerical errors, and a
comparison with the predictions of the background and theory underlying
the techniques used. This section should highlight particular strengths
and/or weaknesses of the methods used.

 
\section{Conclusion}
This section should summarise the results obtained, detail
conclusions reached, suggest future work, and changes that you would make if you repeated the
experiment. This section should in general be short, 100 to 150 words
being typical for most projects.
\par\noindent
If you have opted to have multiple {\bf Theory, Method, Results}
sections, draw all the results together in a {\bf single} conclusion.
\section{References}

Don't forget this section. Detail the relevant references which
should be cited at the correct place in the text of the report. There
are no fixed rules as to how many references are {\it needed}. Generally
the longer the project, and the more background reading you had to do,
the more references will be required. 

When you cite a reference you must give sufficient information. For
example, for a journal article give, {\it Author}, {\it Title of
article},
{\it Journal Name}, {\it Volumn}, {\it Page}, and {\it Year}, 
while for a book give, {\it Author}, {\it Title},
{\it (Editor if there is one)}, {\it Publisher}, and {\it Year}.        

\bibliographystyle{unsrt}   
\bibliography{bibliography}


\appendix
\section{Appendices}

%Material that is useful background to the report, but is not essential,
%or whose inclusion within the report  would detract from its
%structure and readablity, should be included in appendices. Typical
%material could be diagrams of electronic circuits built, specialist
%data tables used to analyse results, details of computer programs
%written for analysis and display of results, photographic plates,
%and, for computational projects, a copy of all written code.
%
%Again be selective. The appendix is {\bf not} an excuse for you to add every
%last detail and piece of data, but should be used to assist the reader
%of the report by supplying additional material. Not all reports require
%appendices and if the report is complete without this additional
%material leave it out.

\subsection{Simulation Code}

\end{document}
